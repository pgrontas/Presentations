\documentclass[10pt,handout]{beamer}

\usepackage{amsmath,amssymb,amsthm,array}
\usepackage{bm}
\usepackage{multirow}
\usepackage{multicol}
\usepackage{algorithm}
\usepackage{hyperref}
\usepackage{algorithmic}
\usepackage[normalem]{ulem}
\usepackage{fontspec}
\usepackage{numprint}

\setmainfont{CMU Serif}
\setsansfont{CMU Sans Serif}

\usetheme{metropolis} 

\setbeamercolor{framesubtitle}{fg=mDarkTeal}
 
\setbeamertemplate{navigation symbols}{}
\title{Αποδείξεις Μηδενικής Γνώσης}
\author{Διαφάνειες: Παναγιώτης Γροντάς - Αλέξανδρος Ζαχαράκης}

\date{04/12/2018}
\defbeamertemplate*{footline}{shadow theme}
{%
  \leavevmode%
  \hbox{
		\begin{beamercolorbox}[wd=.4\paperwidth,ht=2.5ex,dp=1.125ex,leftskip=.3cm,rightskip=.3cm plus1fil]{title in head/foot}%
			\usebeamerfont{title in head/foot} Zero Knowledge  %
		\end{beamercolorbox}
		\begin{beamercolorbox}[wd=.5\paperwidth,ht=2.5ex,dp=1.125ex,leftskip=.3cm,rightskip=.3cm plus1fil]{title in head/foot}%
			\usebeamerfont{title in head/foot} \hfill \insertsection  %
		\end{beamercolorbox}
		\begin{beamercolorbox}[wd=.1\paperwidth,ht=2.5ex,dp=1.125ex,leftskip=.3cm plus1fil,rightskip=.3cm]{author in head/foot}%
			\usebeamerfont{author in head/foot}\insertframenumber\,/\,\inserttotalframenumber
		\end{beamercolorbox}%
  }%
  \vskip0pt%
}

\institute{ΕΜΠ - Κρυπτογραφία (2018-2019)}

 \hypersetup{
  pdfauthor={Panagiotis Grontas},
  pdftitle={ZK},
  colorlinks=true,
  urlcolor=blue,
  pdfborderstyle={/S/U/W 1}	% border style will be underline of width 1pt
}

\setlength{\columnseprule}{0.4pt}
\begin{document}
\newcommand{\xor}{ \oplus }
\newcommand{\msg}{ \mathtt{M} }
\newcommand{\KEY}{ \mathtt{K} }
\newcommand{\CPH}{ \mathtt{C} }
\newcommand{\keygen}{\mathtt{KeyGen}}
\newcommand{\enc}{\mathtt{Encrypt}}
\newcommand{\dec}{\mathtt{Decrypt}}
\newcommand{\sign}{\mathtt{Sign}}
\newcommand{\verify}{\mathtt{Verify}}
\newcommand{\adv}{$\mathcal{A}$}
\newcommand{\Hash}{\mathcal{H} }
\newcommand{\advb}{$\mathcal{B}$}
\newcommand{\chal}{$\mathcal{C}$}
\newcommand{\cs}{$\mathcal{CS}$}
\newcommand{\Zed}{\mathbb{Z}} 
\newcommand{\zns}{\mathbb{Z}^*_n}
\newcommand{\zs}[1]{\mathbb{Z}^*_{#1}}
\newcommand{\prv}{\ensuremath \mathcal{P}\,}
\newcommand{\ver}{\ensuremath \mathcal{V}\,}
\newcommand{\siml}{$\mathcal{S}\,$}
\newcommand{\green}[1]{\textcolor{teal}{#1}}
\newcommand{\Green}[1]{\textcolor{Teal}{#1}}
\newcommand{\ForestGreen}[1]{\textcolor{ForestGreen}{#1}}
\newcommand{\blue}[1]{\textcolor{blue}{#1}}
\newcommand{\magenta}[1]{\textcolor{magenta}{#1}}
\newcommand{\cyan}[1]{\textcolor{cyan}{#1}}

\newcommand{\twopartdef}[4]
{ 
		\begin{cases}
			#1 , #2 \\
			#3 , #4
		\end{cases} 
}
\begin{frame}
\titlepage
\end{frame}

\begin{frame}{Περιεχόμενα}
\begin{itemize}
\item Εισαγωγή
\item Ορισμός - Εφαρμογές στην Θ. Πολυπλοκότητας
\item Σ-πρωτόκολλα
\item Witness Indistinguishable \& Witness Hiding Πρωτόκολλα
\end{itemize}
\end{frame}

\section{Εισαγωγή}
\begin{frame}{Αποδείξεις}
Αποδείξεις στα μαθηματικά
\pause
\begin{itemize}
\item Στόχος: η αλήθεια μας πρότασης
\pause
\item με ενδιάμεσους συλλογισμούς
\pause
\item οι οποίοι δίνουν όμως επιπλέον πληροφορίες
\pause
\end{itemize}
\begin{block}{Πχ. απόδειξη με Αντί-Παράδειγμα}
O 15 δεν είναι πρώτος \\
\pause
...γιατί διαιρείται από το 3 και το 5
\end{block}
\pause
\magenta{Ερώτημα}: Μπορούμε να πειστούμε για την αλήθεια χωρίς διαρροή επιπλέον πληροφοριών - (κέρδος γνώσης);
\end{frame}

\begin{frame}{Εισαγωγή}
\begin{itemize}
\item Shaffi Goldwasser, Silvio Micali και Charles Rackoff, 1985
\pause
\item Διαλογικά συστήματα αποδείξεων
\pause
\begin{itemize}
\item Υπολογισμός ως διάλογος
\pause
\item Prover ($\prv$): Θέλει να αποδείξει ότι μία συμβολοσειρά ανήκει σε μία γλώσσα (complexity style)
\pause
\item Verifier ($\ver$): Θέλει να ελέγξει την απόδειξη
\pause
\begin{itemize}
\item Μια σωστή απόδειξη πείθει τον $\ver$ με πολύ μεγάλη πιθανότητα
\item Μια λάθος απόδειξη πείθει τον $\ver$ με πολύ μικρή πιθανότητα
\end{itemize}
\end{itemize}
\item Απόδειξη μηδενικής γνώσης
\begin{itemize}
\item Ο $\ver$ πείθεται χωρίς να μαθαίνει τίποτε άλλο - κερδίζει γνώση
\end{itemize}
\pause
\end{itemize}
\alert{Μηδενική γνώση:} Ιδιότητα που προστατεύει τον \prv
Πολλές θεωρητικές και πρακτικές εφαρμογές (Βραβείο Turing 2013)
\end{frame}

\begin{frame}{\textit{\href{http://mathoverflow.net/questions/22624/example-of-a-good-zero-knowledge-proof}{Ένα εύκολο παράδειγμα}}}
\begin{small}
\begin{itemize}
\setlength \itemsep{0.01pt}
\item Ο $\ver$ έχει αχρωματoψία
\pause
\item O $\prv$ έχει δύο ταυτόσημες μπάλες, διαφορετικού χρώματος
\pause
\item Μπορεί να πειστεί ο $\ver$ για το ότι οι μπάλες έχουν διαφορετικό χρώμα (\alert{αφού δεν μπορεί να το μάθει});
\pause
\item \green{Ναι}
\begin{itemize}
\item Ο $\prv$ δίνει τις μπάλες στον $\ver$ (\green{commit})
\item Ο $\ver$ κρύβει τις μπάλες πίσω από την πλάτη του (1 ανά χέρι)
\pause
\item Στην \green{τύχη}, αποφασίζει να τις αντιμεταθέσει (ή όχι)
\pause
\item O $\ver$ παρουσιάζει τα χέρια με τις μπάλες στον $\prv$ (\green{challenge})
\pause
\item Ο $\prv$ απαντάει αν άλλαξαν χέρια (\green{response})
\pause
\item Ο $\ver$ αποδέχεται ή όχι
\pause
\item Αν οι μπάλες \alert{δεν} έχουν διαφορετικό χρώμα (κακόβουλος $\prv$): Πιθανότητα απάτης $50\%$
\pause
\item \green{Επανάληψη}: Μείωση πιθανότητας απάτης (πρέπει να μαντέψει σωστά όλες τις φορές)
\end{itemize}
\end{itemize}
\end{small}
\end{frame}


\begin{frame}{Άλλα παραδείγματα}
 
\begin{itemize}
\item \href{http://www.wisdom.weizmann.ac.il/~naor/PAPERS/waldo.pdf}{Where's waldo} \includegraphics[width=.05\textwidth]{waldo.png}\\
\includegraphics[width=.3\textwidth]{waldo-a.jpg} \pause
\includegraphics[width=.3\textwidth]{waldo-b.jpg} \pause
\includegraphics[width=.2\textwidth]{waldo-c.jpg}

\item Η σπηλιά του Alladin
\href{http://pages.cs.wisc.edu/~mkowalcz/628.pdf}{How to explain zero-knowledge protocols to your children}

\item \href{http://blog.computationalcomplexity.org/2006/08/zero-knowledge-sudoku.html}{Γνώση λύσης sudoku}
 
\end{itemize}
 
\end{frame}

\begin{frame}{Εφαρμογές στην κρυπτογραφία}
\begin{itemize}
\item Σχήματα αυθεντικοποίησης αντί για passwords
\begin{itemize}
\item Αντί για κωδικό: Απόδειξη ότι ο χρήστης τον γνωρίζει
\item Αποφεύγεται η μετάδοση και η επεξεργασία
\item Secure Remote Password protocol (SRP  - RFC 2945)
\end{itemize}
\pause
\item Απόδειξη ότι το κρυπτοκείμενο περιέχει μήνυμα συγκεκριμένου τύπου
\pause
\item Ψηφιακές υπογραφές
\item Άντι-malleability
\pause 
\item Γενικά: Απόδειξη ότι παίκτης ακολουθεί κάποιο πρωτόκολλο χωρίς αποκάλυψη ιδιωτικών δεδομένων του
\end{itemize}
\end{frame}

\section{Συστήματα Αποδείξεων Μηδενικής Γνώσης}
\begin{frame}{Διαλογικά Συστήματα Αποδείξεων}
\begin{block}{Συμβολισμός}
\begin{itemize}
\item Γλώσσα $ \mathcal{L} \in \mathtt{NP}$ \pause
\item Πολυωνυμική Mηχανή Turing $\mathcal{M}$ \pause
\item $x \in \mathcal{L} \Leftrightarrow \exists w \in \{0,1\}^{p(|x|)}: M(x,w) = 1$ \pause
\item Δύο μηχανές Turing  $\prv$, $\ver$ \pause
\item $\langle \mathcal{P}(x,w), \mathcal{V}(x) \rangle$ είναι η αλληλεπίδραση μεταξύ  $\prv$, $\ver$ με κοινή (δημόσια είσοδο) το $x$ και ιδιωτική είσοδο του $\prv$ το $w$. \pause
\item $out_\mathcal{V}{\langle \mathcal{P}(x,w), \mathcal{V}(x) \rangle}$ η έξοδος του $\ver$ στο τέλος του πρωτοκόλλου
\end{itemize}
\end{block}
\end{frame}

\begin{frame}{Διαλογικά Συστήματα Αποδείξεων: Παράδειγμα}
\begin{itemize}
\item $\mathcal{L}$ η γλώσσα του προβλήματος του διακριτού λογαρίθμου \pause
\item $x$ ένα στιγμιότυπο του προβλήματος $x=\langle p,g: \langle g \rangle = \mathbb{Z}_p^*, b  \in_R \mathbb{Z}_p^* \rangle$ \pause
\item $w$ ο 'μάρτυρας', δηλ. $a: b = g^a$
\end{itemize}
\end{frame}


\begin{frame}{Διαλογικά Συστήματα Αποδείξεων: Πληρότητα}
Μία απόδειξη μηδενικής γνώσης για την $ \mathcal{L} $ είναι μία αλληλεπίδραση $\langle\mathcal{P}(x,w), \mathcal{V}(x) \rangle$ με τις εξής ιδιότητες: \pause

\begin{block}{\textbf{Πληρότητα - Completeness}}  
Ο \emph{τίμιος} $\prv$, πείθει έναν \emph{τίμιο} $\ver$ με  βεβαιότητα \pause

Αν  $x \in \mathcal{L}$ και $M(x,w) = 1$
\begin{align*}
Pr[out_{\mathcal{V}} \langle \mathcal{P}(x,w), \mathcal{V}(x) \rangle (x)=1] = 1  
\end{align*}
\end{block} 
\end{frame}

\begin{frame}{Διαλογικά Συστήματα Αποδείξεων: Ορθότητα}
\begin{block}{\textbf{Ορθότητα - Soundness}}
Κάθε \emph{κακόβουλος} $\mathcal{P}$ (σμβ. με $\mathcal{P}^*$), δεν μπορεί να πείσει \emph{τίμιο} $\ver$, παρά με αμελητέα πιθανότητα.
\pause
Αν $x \notin \mathcal{L}$ τότε $\forall (\mathcal{P}^*,w^*)$: 
\pause
\begin{align*}
Pr[out_{\mathcal{V}} \langle \mathcal{P}^*(x,w^*), \mathcal{V}(x) \rangle (x)=1] = negl(\lambda)
\end{align*} 
\end{block}
\pause
\textbf{Παρατήρηση: }\\
Proof of Knowledge: O $\mathcal{P}^*$ \alert{δεν} είναι PPT. \\
Argument of Knowledge: O $\mathcal{P}^*$  είναι PPT.
\end{frame}

\begin{frame}{Διαλογικά Συστήματα Αποδείξεων: Μηδενική Γνώση}

\begin{block}{Διαίσθηση}
	O $\ver$ δεν μαθαίνει \alert{τίποτε εκτός από το γεγονός ότι ο ισχυρισμός του $\prv$ είναι αληθής}. \pause
	 
	Ό,τι μπορεί να υπολογίσει ο $\ver$ μετά την συζήτηση με τον $\prv$, μπορεί να το υπολογίσει και \alert{μόνος} του
	
	ή ισοδύναμα με μια συζήτηση με κάποια TM που δεν διαθέτει τον witness \pause
	(προσομοίωση συζήτησης με simulator \siml) 
	 
	(δηλαδή ουσιαστικά χωρίς τη συζήτηση με τον πραγματικό $\prv$)
	\pause
	Άρα: η συζήτηση προσθέτει \emph{μηδενική γνώση}
	
	\end{block}
\end{frame}

\begin{frame}{Διαλογικά Συστήματα Αποδείξεων: Μηδενική Γνώση}

\begin{block}{Ορισμός για \textbf{(Τέλεια) Μηδενική Γνώση:}}
 
\blue{Για κάθε PPT $\mathcal{V}^*$} \magenta{υπάρχει μία PPT \siml}: 
\pause
Αν  $x \in \mathcal{L}$ και $M(x,w) = 1$ οι τυχαίες μεταβλητές 
\pause
\begin{align*}
  out_{\mathcal{V}^*}\langle \mathcal{P}(x,w), \mathcal{V}^*(x) \rangle (x)  \, \, \text{και} \\
  out_{\mathcal{V}^*}\langle \mathcal{S}(x), \mathcal{V}^*(x) \rangle(x)   
\end{align*}

ακολουθούν ακριβώς την ίδια κατανομή.
\end{block}

\alert{κακόβουλος verifier}  προσπαθεί να μάθει το $w$ είτε παθητικά είτε χωρίς να ακολουθεί το πρωτόκολλο
\end{frame}

\begin{frame}{Απόδειξη ιδιότητας ΖΚ: Ο simulator}
{Θεωρητική κατασκευή με πρακτικές εφαρμογές}
\textbf{Δεν διαθέτει τον witness}
\pause
\begin{small}
\begin{itemize}
\setlength\itemsep{0.01em}
\item Προσομοίωση απόδειξης στη θέση του $\prv$ 
\item Αλληλεπιδρά με τον $\ver$
\pause
\item Οι αλληλεπιδράσεις $\langle$\siml,$\ver$$\rangle$ και $\langle$$\prv$,$\ver$$\rangle$ είναι μη διακρίσιμες
\pause
\item Επιτρέπουμε και rewinds:
\pause
\begin{itemize}
\item Αν κάποια στιγμή ο $\ver$ 'ρωτήσει' κάτι που δεν μπορεί να απαντήσει ο \siml τότε stop - rewind
\end{itemize}
\pause
\item Μηδενική γνώση αν ο $\ver$ κάποια στιγμή αποδεχτεί (έστω και με rewinds)
\pause
\item Γιατί: \pause
Δεν μπορεί να ξεχωρίσει τον $\prv$ (που διαθέτει witness) από τον \siml (που δεν διαθέτει) 
\pause
\item \green{Αρκεί ο \siml να παραμείνει PPT}
\pause
\item Συγκεκριμένα: Ένας $\ver$ που εξάγει πληροφορία από τον $\prv$ θα εξάγει την ίδια πληροφορία και από τον \siml (όπου δεν υπάρχει κάτι να εξαχθεί)

\end{itemize}
\end{small}
\end{frame}

\begin{frame}{Σχέση Oρθότητας - Μηδενικής Γνώσης}

O \siml μοιάζει με κακό $\prv^*$ (και οι δύο δεν διαθέτουν τον witness). \\
\medskip

\begin{columns}
\column{0.5\textwidth}

\begin{block}{Ο $\prv^*$}
	\begin{itemize}
		\item Δεν γνωρίζει $w$
		\item Ορθότητα: Δεν πρέπει να πείσει τον $V$
		\item Μπορεί να μην είναι PPT
	\end{itemize}
\end{block}
 
\column{0.5\textwidth}
\begin{block}{Ο \siml}
\begin{itemize}
	\item Δεν γνωρίζει $w$
	\item ΖΚ: Πρέπει να πείσει τον $\ver^*$ με \emph{rewinds}
	\item Πρέπει να είναι PPT
\end{itemize}
\end{block}
\end{columns}

\medskip
\begin{block}{Για τον $\ver$}
\begin{itemize}
\item Στην ορθότητα πρέπει να είναι τίμιος
\item Στην μηδενική γνώση όχι
\end{itemize}
\end{block}

\end{frame}


\begin{frame}{Σύνθεση πρωτοκόλλων μηδενικής γνώσης}

	\begin{block}{Σειριακή}
		Είναι δυνατή η εκτέλεση πολλών πρωτοκόλλων ΖΚ το ένα  μετά το άλλο
		Το αποτέλεσμα ΔΙΑΘΕΤΕΙ ZK
	\end{block}
	\pause
	\begin{block}{Παράλληλη}
	Γενικά \alert{δεν} είναι δυνατή. 

	Η παράλληλη εκτέλεση δύο πρωτοκόλλων ΖΚ δεν παράγει πρωτόκολλο ΖΚ.
	\pause
	{Αιτία - Ιδέα}
	\begin{itemize}
		\item $\prv_1, \prv_2$ (unbounded) zero knowledge provers
		\item $\ver^*$: PPT δεν μπορεί να διακρίνει τις απαντήσεις 
		\item Σε παράλληλη εκτέλεση: Με βάση τις απαντήσεις του $\prv_1$ κατασκευάζει ερωτήσεις για τον $\prv_2$ από τις οποίες εξάγει γνώση για το statement του $\prv_1$		
	\end{itemize}
		
	\end{block}
\end{frame}

\subsection{Παραλλαγές}

\begin{frame}[allowframebreaks]{Παραλλαγές Μηδενικής Γνώσης}
\begin{itemize}
	\item \textbf{Black-Box Zero Knowledge} 
	
	\magenta{$\exists$ PPT \siml} , \blue{$\forall \mathcal{V}^*$}  \\
	$ out_{\mathcal{V}^*} \langle \mathcal{P}(x,w), \mathcal{V}^*(x) \rangle (x) $ και $ out_{\mathcal{V}^*} \langle \mathcal{S}^{ \mathcal{V}^*}(x), \mathcal{V}^*(x) \rangle (x) $ να ακολουθούν ακριβώς την ίδια κατανομή.
	
	Παρατηρήσεις: O \siml 
	\begin{itemize}
	    \item ισχύει για όλους τους $\ver$
	    \item έχει oracle access στον $\ver$
	    \item δηλ. ελέγχει το input, rewind αλλά όχι το output 
	\end{itemize}
	
	\framebreak
	
	\item \textbf{Almost Perfect (Statistical) Zero Knowledge} Οι κατανομές των συζητήσεων με $\prv,\siml$ έχουν αμελητέα στατιστική απόσταση. 
	$\Delta(X,Y) = \frac{1}{2} \sum_{u \in V}|Prob_[X=u] - Prov[Y=u]|=negl(\lambda), \Lambda = |x|$
	
 	\item \textbf{Computational Zero Knowledge} Οι κατανομές των συζητήσεων δεν μπορούν να διαχωριστούν από κάποιον αντίπαλο με πολυωνυμική υπολογιστική ισχύ.
 	
	\framebreak
	
	\item \textbf{Honest Verifier Zero Knowledge}
	\begin{itemize}
	\item Ο $\ver$  είναι τίμιος δηλ:
	\item ακολουθεί το πρωτόκολλο
	\item τα μηνύματα του προέρχονται από την ομοιόμορφη κατανομή - δεν εξαρτώνται από τα μηνύματα του $\prv$ 
	\item μοντελοποιεί και παθητικό αντίπαλο
	\end{itemize}	  
	Πρακτικά: ο \siml παράγει συζητήσεις οι οποίες έχουν ίδια κατανομή με αυθεντικές $\langle \mathcal{P}(x,w), \mathcal{V}(x) \rangle$ 

	\item \textbf{Witness hiding - Witness Indistinguishable proofs}
	\begin{itemize}
		\item WH - δεν μπορεί να γίνει γνωστός ολόκληρος ο μάρτυρας
		\item WI - δεν μπορεί να γίνει διάκριση ποιου μάρτυρα από κάποιες επιλογές
	\end{itemize}

	Ισχύει παράλληλη σύνθεση και έχουν καλύτερη απόδοση
\end{itemize}

\end{frame}

\begin{frame}{Διαφορά ZK - HVZK}
    \begin{block}{... είναι στον $\ver$}
    \begin{itemize}
        \item Σε HVZK:
        \begin{itemize}
            \item Τα μηνύματα του $\ver$ είναι τυχαία 
            \item Μπορούν να προετοιμαστούν εκ των προτέρων  από τον \siml
            \item Άρα o $\ver$ \emph{δεν χρειάζεται} (non interactive)
        \end{itemize} 
        \item Σε ZK:
        \begin{itemize}
            \item Τα μηνύματα του $\ver$ εξαρτώνται από τα μηνύματα του $\prv$
        \end{itemize}
    \end{itemize}    
    \end{block}
\end{frame}

\begin{frame}{Παραλλαγές Ορθότητας}
\begin{block}{Ειδική ορθότητα (special soundness)}
Υπάρχει ένας PPT αλγόριθμος (extractor), $\mathcal{E}$ ο οποίος αν δεχθεί \emph{πολλά} transcripts του πρωτοκόλλου με το ίδιο αρχικό μήνυμα από τον $\prv$ αλλά διαφορετικές προκλήσεις από τον $\ver$ μπορεί να εξάγει τον witness.
\end{block}
\pause
\begin{block}{Θεώρημα}
Ειδική ορθότητα $\Rightarrow$ ορθότητα με πιθανότητα false-positive $\frac{1}{|C|}$ όπου:
$C$: το σύνολο προέλευσης των μηνυμάτων του $\ver$

Ειδική ορθότητα $\Rightarrow$ απόδειξη γνώσης
\end{block}
\end{frame}

\subsection{Αποδείξεις Μηδενικής Γνώσης για Γνωστά Προβλήματα}
\begin{frame}{Graph Isomorphism}
	\begin{block}{Ορισμός}
		Γραφήματα $G_0=(V_0,E_0)$ και $G_1=(V_1,E_1)$ με $|V_0| = |V_1|$ \\ \pause
		Ισχύει ο ισομορφισμός $G_0 \cong G_1$ ανν   
		υπάρχει $\pi :V_0 \rightarrow V_1 $ ώστε $(v_i,v_j) \in E_0 \Leftrightarrow (\pi(v_i),\pi(v_j)) \in E_1$ \pause 
\end{block}
\begin{center}
	\includegraphics[scale=0.5]{gi.png}
\end{center}
\end{frame}

\begin{frame}{GI ZKP}
	Δημόσια είσοδος: Τα γραφήματα $G_0,G_1$\\
	Witness (\prv): $\pi$
	\begin{enumerate}
	   \setlength \itemsep{0.1em}
	   \item $\prv$: εφαρμόζει τυχαία μετάθεση $\pi_1$ στο $V_1$ 
	   \item Προκύπτει γράφημα $F$ ($G_1 \cong F$) το οποίο δημοσιοποιείται στον $\ver$ (δέσμευση) \pause
	   \item $\ver$: Eπιλέγει ένα τυχαίο bit $b$ και το στέλνει στον \prv \pause
	   \item  Αν $b=1$ o \prv δημοσιοποιεί $\phi_b = \pi_1: V_1 \rightarrow V_F$
	   \item  Αν $b=0$ o \prv δημοσιοποιεί $\phi_b = \pi_1 . \pi: V_0 \rightarrow V_F$ ώστε $G_0 \cong F$ \pause
	   \item O $\ver$ δέχεται ανν $\phi_b(G_b) = F$
	   \item Επανάληψη $k$ φορές 
   \end{enumerate}	
\end{frame}

\begin{frame}{GI ZKP: Ιδιότητες}
	Πληρότητα\\
	Αν $\prv, \ver$ έντιμοι και ακολουθούν το πρωτόκολλο τότε σίγουρη αποδοχή
	\begin{itemize}
		\item $b=1: \phi_b(G_b) = \pi_1(G_1) = F$
		\item $b=0: \phi_b(G_b) = \pi_1 . \pi(G_0) = \pi_1(G_1) = F$
	\end{itemize} \pause
	Ορθότητα\\
	Αν $\prv$ δεν έχει $\pi$ ώστε $G_0 \cong G_1$ τότε σε κάθε επανάληψη:
	\begin{itemize}
		\item ο $\ver$ δέχεται με πιθανότητα $\frac{1}{2}$ γιατί o $\prv^*$ δεν μπορεί να γνωρίζει και $\phi_0$ και $\phi_1$
	\end{itemize}   
\end{frame} 

\begin{frame}{GI ZKP: Μηδενική Γνώση}
	Κατασκευή simulator \siml \\ \pause 
	Commitment: Επιλέγει $b'$ και τυχαία μετάθεση $\pi'$ \\ \pause 
	Υπολογίζει $F = \pi'(G_{b'})$

	Challenge: Αν $b=b'$ τότε αποστολή $\pi'$ αλλιώς rewind

	Πιθανότητα αποδοχής σε $k$ επαναλήψεις $2^{-k}$

	Αναμενόμενος χρόνος εκτέλεσης: $T_V \sum_{i=1}^{\infty} 2^{-k} = T_V$ που είναι πολυωνυμικός
\end{frame}

\begin{frame}{3-colorability}
\begin{columns}
\column{0.5\textwidth}
\begin{center}
\includegraphics[scale=0.5]{3cp.jpg}
\end{center}

\column{0.5\textwidth}
\begin{block}{Ορισμός}
Γράφημα $G=(V,E)$ \\ \pause
O $\prv$  γνωρίζει ένα χρωματισμό  $c:V \rightarrow \{ 1,2,3 \}$  \\ \pause
Έγκυρος χρωματισμός: Γειτονικές κορυφές έχουν διαφορετικό χρώμα
$(v_i, v_j) \in E \Rightarrow   c(v_i) \neq c(v_j)$\\
\end{block}
\end{columns}
NP-Complete
\end{frame}
 
\begin{frame}{ZKP for 3-colorability}{Γενική Περιγραφή}
\begin{columns}
\column{0.73\textwidth}
\begin{small}
\begin{enumerate}
 	\setlength \itemsep{0.1em}
	\item $\prv$: επιλέγει μια τυχαία μετάθεση $\pi$ του $\{ 1,2,3 \}$. \pause
	\begin{itemize}
		\item Προκύπτει εναλλακτικός έγκυρος 3 - χρωματισμός $\pi.c$ του $G$. \pause
		\item Χρήση σχήματος δέσμευσης για τον εναλλακτικό χρωματισμό
		\item Yπολογίζει $commit( (\pi.c)(v_i), r_i) \forall v_i \in V$ 
		\item Αποστολή δεσμεύσεων στον $\ver$ \pause
	\end{itemize}
	\item $\ver$: επιλέγει μία τυχαία ακμή $(v_i, v_j) \in E$ και την στέλνει στον $\prv$. \pause
	\item $\prv$: ανοίγει τις δεσμεύσεις - αποκαλύπτει τις τιμές $\pi.c(v_i),\pi.c(v_j)$ και $r_i, r_j$ \pause
	\item $\ver$: ελέγχει αν $\pi.c(v_i) \neq \pi.c(v_j)$ και οι δεσμεύσεις είναι έγκυρες
	\item Επανάληψη
\end{enumerate}
\end{small}
\column{0.27\textwidth}
\begin{center}
\includegraphics[scale=0.6]{3cp2.jpg}
\end{center}
\end{columns}
\end{frame}

 
\begin{frame}{ZKP for 3-colorability: Ιδιότητες (Πληρότητα)} 
\begin{itemize}
\item \textbf{Πληρότητα}\\
Αν ο $c$ είναι έγκυρος χρωματισμός τότε και ο $\pi.c$ είναι έγκυρος χρωματισμός

Το άνοιγμα των δεσμεύσεων θα γίνει αποδεκτό από $\ver$
\end{itemize}
\end{frame}


\begin{frame}{ZKP for 3-colorability: Ιδιότητες (Ορθότητα)} 
\begin{itemize}
\item \textbf{Ορθότητα}\\
Έστω $\mathcal{P}^*$ με μη έγκυρο χρωματισμό για κάποιο γράφημα:

Δηλ. \alert{τουλάχιστον 2 γειτονικές κορυφές με το ίδιο χρώμα}:
\pause

Πιθανότητα ανίχνευσης εξαπάτησης από $\ver$ = Πιθανότητα επιλογής 'κακής' ακμής = $\frac{1}{|E|}$  \\

Πιθανότητα επιτυχούς εξαπάτησης από $\mathcal{P}^*$ = $1-\frac{1}{|E|}$
\pause

Σε $|E|^2$ επαναλήψεις \pause και εφόσον
\begin{center}
$(1+\frac{t}{n})^n \leq e^t$
\end{center}
\pause
\medskip
Πιθανότητα επιτυχίας του  $\mathcal{P}^*$:

\begin{center}
$(1-\frac{1}{|E|})^{|E|^2} \leq e^{-|E|}$ \pause \alert{αμελητέα} ως προς το μέγεθος του γραφήματος
\end{center}

\end{itemize}
\end{frame}

\begin{frame}{ZKP for 3-colorability: Ιδιότητες (Μηδενική Γνώση)} 
\begin{itemize}
\item \textbf{Μηδενική Γνώση} 
\begin{itemize}
\item Χρήση \siml χωρίς γνώση έγκυρου χρωματισμού \pause
\item Ο \siml  επιλέγει τυχαίο χρωματισμό \pause
\item Πιθανότητα επιλογής από $\ver$  ακμής με διαφορετικά χρώματα κορυφών $\frac{2}{3}$ \pause
\item Πιθανότητα επιλογής από $\ver$  ακμής με ίδια χρώματα κορυφών  $\frac{1}{3}$ \pause
\item Αν ο $\ver$  επιλέγει 'κακή' ακμή, rewind (και εκτέλεση από την αρχή) \pause

\item Για $k$ επιτυχείς επιλογές χρειάζονται κατά μέσο όρο $2k$ εκτελέσεις

\end{itemize}
\end{itemize}
\end{frame}

\begin{frame}{ZKP for 3-colorability: Ιδιότητες (Μηδενική Γνώση)} 
Συμπέρασμα: \pause
Ο \siml δεν απαιτεί πολύ περισσότερο χρόνο από έναν $\prv$ με γνώση του $c$ \pause

\alert{Όμως οι συζητήσεις δεν είναι πανομοιότυπες! (Γιατί;)}

\pause 
Τα commitments του $\prv$ είναι έγκυροι χρωματισμοί, ενώ του \siml όχι!
\pause
\begin{block}{Συνέπεια [GMW91]}
Αν υπάρχουν computationally hiding bit commitment schemes τότε όλο το NP έχει αποδείξεις μηδενικής γνώσης (black box computational)
\end{block}

\end{frame}

\section{Σ-πρωτόκολλα}

\begin{frame}{Σ-πρωτόκολλα}
Χαλάρωση ZK με τίμιο verifier

\begin{block}{Ορισμός}
Ένα πρωτόκολλο 3 γύρων με honest verifier και special soundness
\begin{enumerate}
	\item \textbf{Commit} O $\prv$ δεσμεύεται σε μία τιμή. \pause 
	\item \textbf{Challenge} Ο $\ver$ διαλέγει μία τυχαία πρόκληση. Εφόσον είναι τίμιος θεωρούμε ότι η πιθανότητα επιλογής πρόκλησης είναι ομοιόμορφα κατανεμημένη. \pause
	\item \textbf{Response} O $\prv$ απαντάει χρησιμοποιώντας τη δέσμευση, το μυστικό και την τυχαία τιμή. \pause
\end{enumerate}
\end{block}

\begin{block}{Special Soundness}
Δύο εκτελέσεις του πρωτοκόλλου με το ίδιο commitment, οδηγούν στην αποκάλυψη του witness
\end{block}

\end{frame}

\subsection{Schnorr}
\begin{frame}[allowframebreaks]{Γνώση DLOG:Το πρωτόκολλο του Schnorr}
\begin{block}{Γνωστά Στοιχεία}
\begin{itemize}
\item \textbf{Δημόσια:} Γεννήτορας $g$ μιας (υπό)ομάδας τάξης $q$ του $\zs{p}$ με δύσκολο DLP και στοιχείο $h \in \zs{p}$ 
\item \textbf{Ιδιωτικά:} O $\prv$ έχει ένα witness $x \in \zs{q}$ ώστε $h = g^x \pmod{p}$
\end{itemize}
\end{block}

\begin{block}{Στόχος}
Απόδειξη κατοχής του $x$ χωρίς να αποκαλυφθεί.
\end{block}

\begin{block}{Συμβολισμός Camenisch-Stadler}
$PoK \{(x): g^x = h \pmod{p}, h,g \in_R \mathbb{Z}_p^* \}$ 
\end{block}

\framebreak

\begin{columns}
\column{0.5\textwidth}
\begin{small}
\begin{itemize}
\item  \textbf{Commit ($\prv$ $\rightarrow$ $\ver$):} 
\begin{itemize}
\item Τυχαία επιλογή $t \in_R \zs{q}$ 
\item Yπολογισμός $y = g^t \bmod{p}$. 
\item Αποστολή $y$  στον $\ver$. 
\end{itemize}
\item \textbf{Challenge ($\ver$ $\rightarrow$ $\prv$):} \\  Τυχαία επιλογή και αποστολή $c \in_R \zs{q}$
\item \textbf{Response ($\prv$ $\rightarrow$ $\ver$):} \\   O $\prv$ υπολογίζει το $s=t+cx \bmod{q}$ και το στέλνει στον $\ver$
\item  Ο $\ver$ αποδέχεται αν\\ $g^s = yh^c \pmod{p}$
\end{itemize}
\end{small}
\column{0.5\textwidth}
\begin{figure}
\includegraphics[width=1\textwidth]{schnorr.png}
\end{figure}
\end{columns}
\end{frame}

\begin{frame}{Πρωτόκολλο Schnorr: Πληρότητα}
 
\begin{itemize}
\item \textbf{Πληρότητα}\\

\pause
\begin{center}
$g^s = g^{t+cx} = g^t g^{cx} = yh^c \pmod{p}$
\end{center}



\end{itemize}
\end{frame}


\begin{frame}{Πρωτόκολλο Schnorr: Ορθότητα}
\begin{itemize}
\item \textbf{Ορθότητα} 
Πιθανότητα ο $\prv$$^*$ να ξεγελάσει τίμιο verifier: $\frac{1}{q}$ - αμελητέα - επανάληψη για μεγαλύτερη σιγουριά
\pause
\item \textbf{Special soundness}\\
Έστω 2 επιτυχείς εκτελέσεις του πρωτοκόλλου $(y,c,s)$ και $(y,c',s')$
\pause
\begin{align*}
 g^s = yh^c  \text{ και }  g^{s'} = yh^{c'}  \Rightarrow  g^s h^{-c}   = g^{s'} h^{-c'}  \Rightarrow \\
 g^{s-xc} = g^{s'-xc'} \Rightarrow  s-xc = s'-xc' \Rightarrow \\
 x = \frac{c'-c}{s-s}
\end{align*}
\pause
Αφού o \prv μπορεί να απαντήσει 2 τέτοιες ερωτήσεις ξέρει το DLOG
(ορθότητα και γνώση)
\end{itemize}
\end{frame}


\begin{frame}{Πρωτόκολλο Schnorr: HVZK}
\begin{itemize}
\item Διαθέτει \green{Honest Verifier Zero Knowledge}

Έστω  \siml που δεν γνωρίζει το $x$ και τίμιος $\ver$ 
\pause
\begin{itemize}
\item Αρχικά o \siml δεσμεύεται κανονικά στο $y=g^t, t \in_R \zs{q}$
\pause
\item Ο $\ver$ επιλέγει $c \in_R \zs{q}$
\pause
\item Αν ο \siml μπορεί να απαντήσει (αμελητέα πιθανότητα) το πρωτόκολλο συνεχίζει κανονικά \pause
\item Αλλιώς γίνεται rewind ο $\ver$ (ίδιo random tape) \pause
\item Στη δεύτερη εκτέλεση o \siml δεσμεύεται στο $y=g^t h^{-c}, t \in_R \zs{q}$ \pause
\item Ο $\ver$ επιλέγει ίδιο $c \in_R \zs{q}$ (ίδιο random tape) \pause
\item O \siml στέλνει $s=t$ \pause
\item Ο $\ver$ θα δεχτεί αφού \\
$yh^{c} = g^t  h^{-c} h^{c} = g^t = g^s$ \\
\end{itemize}
\end{itemize}
\pause
 
\begin{block}{Δηλαδή:}
Η συζήτηση $(t \in_R \mathbb{Z}_q; g^t h^{-c}   , c \in_R \mathbb{Z}_q  , t )$ 
 και η $(t,c \in_R \mathbb{Z}_q;  g^t  , c  , t+xc  )$
 ακολουθούν την ίδια κατανομή
 \end{block}
\end{frame}

\begin{frame}{Πρωτόκολλο Schnorr: ZK}

\textbf{Μηδενική Γνώση}: \alert{Δε διαθέτει}
\pause
\begin{itemize}
\item Ένας cheating verifier δε διαλέγει τυχαία \pause
\item Bασίζει κάθε challenge στο προηγούμενο commitment του \siml \pause
\item Στη simulated εκτέλεση δεν θα επιλέξει το ίδιο challenge \pause
\item Αμελητέα πιθανότητα να μπορεί να απαντηθεί από τον \siml \pause
\end{itemize}

Ενίσχυση για μηδενική γνώση: \pause

\begin{itemize}
\item Προσθήκη δέσμευσης από τον $\ver$ στην τυχαιότητα \emph{πριν} το πρώτο μήνυμα του $\prv$ ή \pause
\item Challenge space $\{ 0 , 1 \} $ (γιατί;) \pause
\item Ο $\ver$ έχει δύο επιλογές μόνο για επιλογή πρόκλησης. 
\item Αν αλλάξει, ο \siml μπορεί να προετοιμαστεί και για τις δύο περιπτώσεις.
\end{itemize} 

\end{frame}

\subsection{Chaum-Pedersen}
\begin{frame}[allowframebreaks]{Ισότητα DLOG:Το πρωτόκολλο Chaum Pedersen}
\begin{block}{Γνωστά Στοιχεία}
\begin{itemize}
\item \textbf{Δημόσια:} Γεννήτορες $g_1, g_2$ μιας (υπό)ομάδας τάξης $q$ του $\zs{p}$ με δύσκολο DLP και 2 στοιχεία $h_1, h_2 \in \zs{p}$ 
\item \textbf{Ιδιωτικά:} O $\prv$ έχει ένα witness $x \in \mathbb{Z}_q$  ώστε $h_1 = g_{1}^{x} \bmod{p}$, $h_2 = g_{2}^{x} \bmod{p}$
\end{itemize}
\end{block}

\begin{block}{Στόχος}
Απόδειξη γνώσης του $x$ χωρίς να αποκαλυφθεί

Απόδειξη ισότητας διακριτών λογαρίθμων
\end{block}

\begin{small}
$PoK \{(x): h_1 = g_1^x \pmod{p} \wedge h_2 = g_2^x \pmod{p},  h_1,g_1,h_2,g_2 \in_R \mathbb{Z}_p^* \}$
\end{small}

\begin{columns}

\column{0.5\textwidth}
\begin{small}
\begin{itemize}
\item  \textbf{Commit:}
\begin{itemize}
\item Ο $\prv$  διαλέγει $t \in_R \mathbb{Z}_q$
\item Yπολογίζει $y_1 = g_{1}^{t} \bmod{p}$\\ $y_2 = g_{2}^{t} \bmod{p}$
\item Αποστέλλει $y_1, y_2$  στον $\ver$ 
\end{itemize}
\item  \textbf{Challenge:}\\  
Ο $\ver$ διαλέγει και αποστέλλει $c \in_R \mathbb{Z}_q$ 
\item  \textbf{Response:}\\   
Ο $\prv$ υπολογίζει $s = t+cx \bmod{q}$ και το στέλνει στον $\ver$
\end{itemize} 
\end{small}
\column{0.5\textwidth}
\begin{figure}
\includegraphics[width=1\textwidth]{chaumpedersen.png} 
\end{figure}
\end{columns}

\begin{center}
Ο $\ver$ δέχεται αν  $g^s_1 = y_1h_{1}^c \pmod{p}$ και  $g^s_2 = y_2h_{2}^c \pmod{p}$ 
\end{center}


\end{frame}

\begin{frame}[allowframebreaks]{Ιδιότητες Chaum-Pedersen}
\begin{itemize}
\item \textbf{Πληρότητα}\\
Αν $h_1 = g_1^x$ και $h_2 = g_2^x$ τότε:
\begin{align*}
g_1^s = g_1^{t+xc} = y_1h_1^c \\
g_2^s = g_2^{t+xc} = y_2h_2^c
\end{align*}
\item \textbf{Special soundness}\\
Έστω δύο αποδεκτά transcripts με το ίδιο commitment $((y_1,y_2),c,s)$ και $((y_1,y_2),c',s')$
\begin{align*}
g_1^s =  y_1h_1^c \text{  και  } g_1^{s'} =  y_1h_1^{c'} \Rightarrow g_1^s h_1^{-c} = g_1^{s'} h_1^{-c'} \\
g_2^s =  y_2h_2^c \text{  και  } g_2^{s'} =  y_2h_2^{c'} \Rightarrow g_2^s h_2^{-c} = g_2^{s'} h_2^{-c'}
\end{align*}
Όπως σε Schnorr $x = \frac{s-s'}{c'-c}$
\framebreak
\item \textbf{Honest verifier zero knowledge}\\
Πραγματικό transcript με $c \in_R \mathbb{Z}_q$: 
\begin{center}
$(t \in_R \mathbb{Z}_q;(g_1^t,g_2^t), \quad c \in_R \mathbb{Z}_q,\quad t+xc \bmod{q})$
\end{center}
Simulated transcript με $c \in_R \mathbb{Z}_q$: 
\begin{center}
$(t,c \in_R \mathbb{Z}_q;(g_1^t h_1^{-c},g_2^th_2^{-c}), \quad c, \quad t)$
\end{center}
Ίδιες κατανομές αν $x=log_{g_1}{h_1}=log_{g_2}{h_2}$
\end{itemize}
\end{frame}

\begin{frame}{Εφαρμογές}
\begin{small}
\begin{block}{Έλεγχος για τριάδες DH}
Η τριάδα $(g^a,g^b,g^c)$ είναι τριάδα DH (δηλ. $g^c = g^{ab}$) 
\end{block}
\pause
Εκτελούμε $\mathtt{CP}(g_1 = g,g_2 = g^b,h_1 = g^a, h_2 = g^{ab} = {g^b}^a)$ με witness $a$
\begin{block}{Εγκυρότητα κρυπτογράφησης El-Gamal}
Δίνεται ένα ζεύγος στοιχείων του $\zs{p}$ τα $(c_1,c_2)$. 

Να δειχθεί ότι αποτελούν έγκυρη κρυπτογράφηση ενός μηνύματος $m$.
\end{block}
\pause
Αν είναι έγκυρη τότε πρέπει
\begin{align*}
(c_1,c_2) = (g^r, m \cdot h^r)
\end{align*}

Ισοδύναμα: 
\begin{align*}
log_g c_1 = log_h (\frac{c_2}{m}) 
\end{align*}
δηλ. ότι ο $\prv$ είναι γνώστης της τυχαιότητας
\end{small} 
\end{frame}

\subsection{Σύνθεση $\Sigma$ πρωτοκόλλων}
\begin{frame}[allowframebreaks]{Σύνθεση $\Sigma$ πρωτοκόλλων}
\begin{block}{Θέωρημα} 
Τα $\Sigma$ πρωτόκολλα διατηρούν τις ιδιότητες τους αν συνδυαστούν με τις παρακάτω σχέσεις:
\end{block}
\begin{itemize}
\item $\mathtt{AND}$
\begin{itemize}
\item O $\prv$ γνωρίζει 2 διαφορετικά $w$ για διαφορετικές σχέσεις.
\item Απόδειξη: 2 παράλληλες εκτελέσεις του  $\Sigma$ πρωτόκολλου με ίδιο challenge
\end{itemize}
\begin{figure}
\centering
\includegraphics[width=0.6\textwidth]{schnorrand.png}
\end{figure}
\framebreak
\item Batch-AND\\
Μαζική επαλήθευση πολλαπλών σχέσεων με ένα πρωτόκολλο.
Για παράδειγμα: \\ 
$(g^a, g^b, g^{ab})$ ΚΑΙ  $(g^c, g^d, g^{cd})$ είναι τριάδες DH\\
Μπορώ να εκτελέσω το Chaum Pedersen για $(g^{ac}, g^{bd}, g^{abcd})$
\item $\mathtt{EQ}$
\begin{itemize}
\item O $\prv$ γνωρίζει τον ίδιο $w$ για διαφορετικές σχέσεις.
\item Chaum Pedersen
\end{itemize}
\item $\mathtt{OR}$
\begin{itemize}
\item O $\prv$ γνωρίζει \emph{κάποιο} $w$ για διαφορετικές σχέσεις.
\item Εφαρμογή: Απόδειξη ότι ο $w$ ανήκει σε ένα σύνολο
\end{itemize}
\end{itemize} 
\end{frame}

\begin{frame}{Γενικευμένη κατασκευή αποδείξεων OR}
\begin{itemize}
	\item Έστω $W = \{w_1, ..., w_n\}$ οι εναλλακτικοί μάρτυρες
	\pause
	\item Για αυτόν που κατέχει ο $\prv$ ακολουθεί το πρωτόκολλο
	\pause
	\item Για τους υπόλοιπους ο $\prv$ καλεί τον \siml ο οποίος υπολογίζει τις δεσμεύσεις που θα έκαναν τον $\ver$ να δεχθεί σε μία προσομοιωμένη συζήτηση
	\pause
	\begin{itemize}
		\item \alert{Πρόβλημα:} O \siml δεν ξέρει το challenge
		\item \green{Λύση:} Το επιλέγει τυχαία
	\end{itemize} 
	\item Όλες οι δεσμεύσεις αποστέλλονται στον $\ver$ 
	\pause
	\item Ο τελευταίος απαντάει με \emph{μία} τυχαία πρόκληση
	\pause
	\item O $\prv$ ερμηνεύει την πρόκληση ως ένα μυστικό που πρέπει να χωριστεί
	\pause
	\item Κάθε μερίδιο θα χρησιμοποιείται στις απαντήσεις του $\prv$ στο στάδιο Response
	\pause
	\item Ο $\ver$ αποδέχεται αν όλες τις απαντήσεις που έλαβε στο τελευταίο βήμα είναι έγκυρες.
\end{itemize}
\end{frame}

\begin{frame}{OR-Schnorr}{$PoK \{(x_1,x_2): h_1 = g_1^{x_1} \pmod{p} \vee h_2 = g_2^{x_2} \pmod{p}$ \} }
Υποθέτουμε ότι ο $\prv$ ξέρει το $x_1$
\begin{figure}
	\centering
	\includegraphics[width=0.8\textwidth]{schnorrwid.png}
\end{figure}
\end{frame}

%\begin{frame}{Γενίκευση}
%\setlength \itemsep{0.01pt}
%\begin{itemize}
%\item Γνώση $k$ από $n$ witnesses
%\item Χρήση $(n-k,n)$ threshold secret sharing
%\item Πώς:
%\begin{itemize}
%\item WLOG: Ο $\prv$ δεν γνωρίζει τους $n-k$ πρώτους, γνωρίζει τους %υπόλοιπους $k$
%\item Commit: Επιλογή $n-k$ τιμών $c_i$ για χρήση στον \siml
%\item Challenge: O $\ver$ στέλνει το $c$ (το μυστικό που πρέπει να %μοιραστεί)
%\item Response:
%\begin{itemize}
%\item Ο $\prv$ επιλέγει βρίσκει πολυώνυμο $p$ που περνάει από τα %$\{(i, c_i)\}_{i=1}^{n-k} \cup (0,c)$
%\item Υπολογίζει τα $\{c_i = p(i) \}_{i=n-k+1}^n$ τα οποία %χρησιμοποιεί στο πρωτόκολλο
%\item Αποστέλει όλα τα $ \{ c_i, s_i \}_{i=1}^n$
%\item O $\ver$ ελέγχει τις σχέσεις και αν μπορεί να γίνει η %ανακατασκευή του $c$
%\end{itemize} 
%\end{itemize}
%\end{itemize}
%\textbf{Επέκταση Σύνθεσης}: Για οποιοδήποτε σύνθετη μονότονη λογική %πρόταση (AND,OR)
%\end{frame}

\subsection{Μη διαλογικές αποδείξεις}
\begin{frame}{Μη διαλογικές αποδείξεις}
\begin{block}{Ερώτηση}
Μπορούμε να καταργήσουμε τον $\ver$ ;
\end{block}

O $\prv$ παράγει την απόδειξη μόνος του

Η απόδειξη είναι επαληθεύσιμη από οποιονδήποτε

\begin{block}{Common Reference String}
Μία ομοιόμορφα επιλεγμένη ακολουθία bits (από κάποια έμπιστη οντότητα) ως κοινή είσοδος σε $\prv,\ver$\\
Χρησιμεύει για την επιλογή των μηνυμάτων που ανταλλάσσονται
\end{block}

\begin{block}{Μετασχηματισμός Fiat Shamir}
Αντικατάσταση της τυχαίας πρόκλησης με το αποτέλεσμα μιας ψευδοτυχαίας συνάρτησης με είσοδο τη δέσμευση (τουλάχιστον)

Συνήθως συνάρτηση σύνοψης - $\mathcal{H}$ (τυχαίο μαντείο)
\end{block}
\end{frame}

\begin{frame}{Non-interactive Schnorr}
\begin{block}{Γνωστά Στοιχεία}
\begin{itemize}
\item \textbf{Δημόσια:} Γεννήτορας $g$ μιας (υπό)ομάδας τάξης $q$ του $\zs{p}$ με δύσκολο DLP και στοιχείο $h \in \zs{p}$ 
\item \textbf{Ιδιωτικά:} O $\prv$ έχει ένα witness $x \in \zs{q}$ ώστε $h = g^x \bmod{p}$
\end{itemize}
\end{block}
\pause
O  $\prv$:
\begin{itemize}
\item Τυχαία επιλογή $t \in_R \mathbb{Z}_{q}$,
\pause
\item Yπολογισμός $y = g^t \bmod{p}$
\pause
\item Υπολογισμός $c = \mathcal{H}(y)$ όπου $\mathcal{H}$ είναι μια συνάρτηση σύνοψης που δίνει τιμές στο $\mathbb{Z}_{q}$
\pause

\item Υπολογισμός $s=t+cx \bmod{q}$
\pause
\item Δημοσιοποίηση του $(h,c,s)$
\pause
\item Επαλήθευση (από οποιονδήποτε) $c = \mathcal{H} (g^s h^{-c})$
\end{itemize}
\end{frame} 

\section{Witness Indistinguishable - Witness Hiding Protocols}

\begin{frame}{Witness Indistinguishability \& Witness Hiding}
	Χαλάρωση ΖΚ για βελτίωση απόδοσης και composability 

	Υποθέτουμε cheating verifier $\ver^*$
	\begin{itemize}
	  \item Ορίζουμε ως $W(x)=\{w : R(x,w)=1 \}$ \pause
	  \item Στις αποδείξεις γνώσης ο $\prv$ θέλει να πείσει τον $\ver$ ότι ξέρει έναν μάρτυρα $w\in W(x)$. \pause
	  \item ZK: O $\ver^*$ δεν μαθαίνει \emph{οτιδήποτε} για το $w$. \pause
	  \item WH: Ο $\ver^*$ δεν μαθαίνει \emph{ολόκληρο} $w \in W(x)$. \pause
	  \item WI: Ο $\ver^*$ δεν μαθαίνει τίποτα για \emph{ποιο} $w \in W(x)$ ξέρει ο $\prv$. \pause
	\end{itemize}

	Σχέση \pause
	\begin{itemize}
		\item ZK $\rightarrow$ WH και ZK $\rightarrow$ WΙ (όχι όμως αντίστροφα)
		\item HVZK $\rightarrow$ WI
		\item Υπο συνθήκες WI $\rightarrow$ WH 
		\item WΗ $\nrightarrow$ WI 
	\end{itemize}
\end{frame}
  
\begin{frame}{Witness Indistinguishability}
\begin{itemize}
	\item Πολλά μυστικά κλειδιά αντιστοιχούν στο ίδιο δημόσιο κλειδί. \pause
	\item Αποδείξεις με διαφορετικά κλειδιά είναι μη διακρίσιμες. \pause
	\item Γνώση δύο κλειδιών οδηγούν σε εξαγωγή ενός μυστικού. 
\end{itemize}
	\pause
\begin{block}{Ορισμός}
	Ένα διαλογικό σύστημα αποδείξεων είναι WI αν $\forall \ver^*$ ισχύει
	\[
			\{ \langle \prv(w), \ver^*(z) \rangle(x)\}_{x\in L,w\in W(x)} \stackrel{\ }{\equiv}
			\{ \langle \prv(w'), \ver^*(z) \rangle(x) \}_{x\in L,w'\in W(x)}
	\]
\end{block}

\end{frame}
  
\begin{frame}
\frametitle{Αναπαράσταση στοιχείου σε ομάδα}
\begin{block}{Ορισμός}
	Έστω $\mathbb{G}$ ομάδα τάξης $q$ και $g_1,g_2\in \mathbb{G}$. Αναπαράσταση του $h\in \mathbb{G}$ ως προς $g_1,g_2$ ονομάζεται κάθε ζεύγος $x_1,x_2\in \mathbb{Z}_q$ τέτοιο ώστε $h=g_1^{x_1}g_2^{x_2}$.
\end{block}
\pause
\alert{Αν ξέρω δύο αναπαραστάσεις του $h$ ως προς $g_1,g_2$ τότε ξέρω διακριτό λογάριθμο του $g_2$ ως προς $g_1$ (βλ. Pedersen commitments)}
\end{frame}
  
  
\begin{frame}\frametitle{Πρωτόκολλο Okamoto Schnorr: WI Proof of Knowledge of Representation}
$PoK \big\{(x_1,x_2): h=g_1^{x_1}g_2^{x_2}, \mathbb{G},q, \, g_1,g_2,h \in \mathbb{G}, \big\}$ \pause 
\begin{itemize}
	\item \textbf{$\prv$:} 
	$r_1,r_2\leftarrow_R \mathbb{Z}_q$;\;   \\
	$a \leftarrow g_1^{r_1}g_2^{r_2}$;\;    \\ 
	Στέλνει $a$. \pause 
	\item \textbf{$\ver$:} 
	$c\leftarrow_R \mathbb{Z}_q$;\;    \\
	Στέλνει $c$. \pause 
	\item \textbf{$\prv$:} 
	$s_1=r_1+x_1c$;\; $s_2=r_2+x_2c$; \\ 
	Στέλνει $s_1,s_2$. \pause 
	\item \textbf{$\ver$:} 
	Αποδέχεται αν $g_1^{s_1}g_2^{s_2}=ah^{c}$.
\end{itemize}
\end{frame}
	 
\begin{frame}\frametitle{Πρωτόκολλο Okamoto Schnorr:Ιδιότητες}	 
\textbf{Ιδιότητες}
Πληρότητα και Ειδική Ορθότητα προφανείς.\pause 

\textbf{WI:} Έστω $h=g_1^{x_1}g_2^{x_2}=g_1^{x_1'}g_2^{x_2'}$ \\
	Τότε \[g_1^{x_1-x_1'}g_2^{x_2-x_2'}=hh^{-1}=1\] \pause 
	Για κάθε transcript $(a,c,s_1,s_2)$ με witness $x_1,x_2$ και τυχαιότητα $r_1, r_2$ στο πρώτο βήμα υπάρχουν $r_1',r_2'$ που δίνουν ακριβώς την ίδια συζήτηση για $x_1',x_2'$. Πράγματι: \pause 
	\begin{align*}
		r_1' &= r_1 + c(x_1 - x_1') \\
		r_2' &= r_2+c(x_2-x_2') \\
		a' &= g_1^{r_1'}g_2^{r_2'} = g_1^{r_1+c(x_1-x_1')}g_2^{r_2+c(x_2-x_2')}=\\
		  &=g_1^{r_1}g_2^{r_2} g_1^{c(x_1-x_1')} g_2^{c(x_2-x_2')}=\\
		  &=a
	\end{align*} 
 
\end{frame}

 

\section{Πηγές}
\begin{frame}[allowframebreaks]{Βιβλιογραφία}
\begin{tiny}
\begin{enumerate}
\item \href{http://hdl.handle.net/11419/5439}{Παγουρτζής, Α., Ζάχος, Ε., ΓΠ, 2015. Υπολογιστική κρυπτογραφία. [ηλεκτρ. βιβλ.] Αθήνα:Σύνδεσμος Ελληνικών Ακαδημαϊκών Βιβλιοθηκών}
\item Jonathan Katz and Yehuda Lindell. Introduction to Modern Cryptography. Chapman and Hall/CRC, 2007
\item Oded Goldreich, The Foundations of Cryptography - Volume 1,  Cambridge University Press, 2001
\item Paar, Christof, and Jan Pelzl. Understanding cryptography: a textbook for students and practitioners. Springer Science-Business Media, 2009.
\item Kiayias, Aggelos  \href{http://crypto.di.uoa.gr/class/Kryptographia/Semeioseis_files/Cryptograph_Primitives_and_Protocols.pdf}{Cryptography primitives and protocols}, UoA, 2015
\item \href{http://goo.gl/b75I29}{Nigel Smart. Introduction to cryptography}
\item Berry Schoenmakers. \href{http://www.win.tue.nl/~berry/2WC13/}{Cryptographic protocols}, 2015. 
\item  D. Chaum and T. P. Pedersen. Wallet databases with observers.  CRYPTO ’92.
\item U. Feige and A. Shamir. 1990. Witness indistinguishable and witness hiding protocols. In STOC '90.
\item R. Cramer, I. Damgard, and B. Schoenmakers. Proofs of partial knowledge and simplified design of witness hiding protocols. In CRYPTO ’94.
\item A. Fiat and A. Shamir. How to prove yourself: practical solutions to identification and signature problems. CRYPTO ’86.
\item O.Goldreich,S.Micali, and A.Wigderson. Proofs that yield nothing but their validity or all languages in np have zero-knowledge proof systems. J. ACM, 38(3):690–728, July 1991.
\item S Goldwasser, S Micali, and C Rackoff. The knowledge complexity of interactive proof-systems. STOC ’85
\item  Jean-Jacques Quisquater, Louis Guillou, Marie Annick, and Tom Berson. 1989. \href{http://pages.cs.wisc.edu/~mkowalcz/628.pdf}{How to explain zero-knowledge protocols to your children}. CRYPTO '89
\item Mike Rosulek, \href{http://web.engr.oregonstate.edu/~rosulekm/pubs/zk-waldo-talk.pdf}{Zero-Knoweldge Proofs, with applications to Sudoku and Where’s Waldo}
\item  C.P. Schnorr. Efficient signature generation by smart cards. Journal of Cryptology, 4(3):161–174, 1991
\item Online Lectures by \href{http://www.cs.jhu.edu/~susan/600.641/}{Susan Hohenberger},  \href{http://www.cs.cornell.edu/courses/cs6830/2014fa/}{Rafael Pass}
\item  Matthew Green, \href{http://blog.cryptographyengineering.com/2014/11/zero-knowledge-proofs-illustrated-primer.html}{Zero knowledge proofs: An illustrated primer}
\item Jeremy Kuhn \href{https://jeremykun.com/2016/07/05/zero-knowledge-proofs-a-primer/}{Zero Knowledge Proofs — A Primer}
\end{enumerate}
\end{tiny}
\end{frame}

\end{document}
