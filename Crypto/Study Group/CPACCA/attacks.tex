\documentclass{beamer}

\usepackage[normalem]{ulem}
\usepackage{amsmath,amssymb,amsthm,array}
\usepackage{bm}
\usepackage{textcomp}
\usepackage{xltxtra}
\usepackage{multirow}
\usepackage{multicol}
\usepackage{algorithm}
\usepackage{algorithmic}
\usepackage{listings}
\usepackage[square, numbers, comma, sort&compress]{natbib}

\usetheme{CambridgeUS}
\usecolortheme{dolphin}
\setmainfont[Mapping=TeX-text]{GFS Neohellenic}
\usefonttheme{serif}
\setbeamertemplate{navigation symbols}{}

\title{Επιθέσεις και Ασφάλεια Κρυπτοσυστημάτων}

\defbeamertemplate*{footline}{shadow theme}
{%
  \leavevmode%
  \hbox{\begin{beamercolorbox}[wd=.5\paperwidth,ht=2.5ex,dp=1.125ex,leftskip=.3cm plus1fil,rightskip=.3cm]{author in head/foot}%
    \usebeamerfont{author in head/foot}\insertframenumber\,/\,\inserttotalframenumber\hfill\insertshortauthor (\insertshortinstitute)
  \end{beamercolorbox}%
  \begin{beamercolorbox}[wd=.5\paperwidth,ht=2.5ex,dp=1.125ex,leftskip=.3cm,rightskip=.3cm plus1fil]{title in head/foot}%
    \usebeamerfont{title in head/foot}\insertshorttitle%
  \end{beamercolorbox}}%
  \vskip0pt%
}

\setlength{\columnseprule}{0.4pt}
\begin{document}
\newcommand{\zns}[1]{ \mathbb{Z}_{#1}^* }
\newcommand{\zn}[1]{ \mathbb{Z}_{#1}}
\newcommand{\md}[1]{\quad (mod \, {#1})}
\newcommand{\adv}{\mathcal{A}}
 
\begin{frame}
\titlepage
\end{frame}

\begin{frame}{Επιθέσεις ενεργητικού αντιπάλου $\adv$} 

\begin{block}{Chosen Plaintext Attack}
\begin{itemize}
\item Ικανότητα: Ο $\adv$ μπορεί να κρυπτογραφεί μηνύματα της αρεσκείας του
\item Στόχος: Ο $\adv$ θέλει να μάθει την αποκρυπτογράφηση ενός κρυπτοκειμένου
\end{itemize}
\end{block}

\begin{block}{Chosen Ciphertext Attack}
\begin{itemize}
\item Ικανότητα: Ο $\adv$ μπορεί να κρυπτογραφεί μηνύματα της αρεσκείας του
\item Ικανότητα: Ο $\adv$ μπορεί να αποκρυπτογραφεί κάποια μηνύματα της αρεσκείας του
\item Στόχος: Ο $\adv$ θέλει να μάθει την αποκρυπτογράφηση ενός συγκεκριμένου διαφορετικού μηνύματος
\end{itemize}
\end{block}
\end{frame}

\begin{frame}{Indistinguishability under Chosen Plaintext Attack (IND-CPA)}

\begin{block}{CPA Game}
\begin{itemize}
\item Δημιουργία ζεύγους κλειδιών (PK,SK)
\item Δημοσίευση PK
\item Ο $\adv$ μπορεί να κρυπτογραφεί πολυωνυμικό πλήθος μηνυμάτων
\item Τελικά υποβάλλει δύο μηνύματα $ M_0, M_1 $  στο σύστημα
\item Το σύστημα διαλέγει τυχαία 1 bit $b$ και αποστέλλει το $C=Enc(M_b)$ στον $\adv$
\item Ο $\adv$ μπορεί να συνεχίσει να κρυπτογραφεί πολυωνυμικό πλήθος μηνυμάτων και να κάνει οποιονδηποτε υπολογισμό θέλει.
\item Τελικά μαντεύει το $b$
\end{itemize}
\end{block}

\begin{block}{Ορισμός ασφάλειας}
Το κρυπτοσύστημα έχει την ιδιότητα IND-CPA αν κάθε PPT $\adv$ έχει αμελητέο πλεονέκτημα στον υπολογισμό του $b$ από το να μαντέψει τυχαία.
\end{block}

\end{frame}


\begin{frame}[allowframebreaks]{Indistinguishability under Chosen Ciphertext Attack (IND-CCA)}

CCA Game

\begin{itemize}
\item Δημιουργία ζεύγους κλειδιών (PK,SK)
\item Δημοσίευση PK
\item Ο $\adv$ μπορεί να κρυπτογραφεί πολυωνυμικό πλήθος μηνυμάτων
\item Ο $\adv$ χρησιμοποιεί το σύστημα ως \textit{decryption oracle} και μπορει να αποκρυπτογραφήσει συγκεκριμένα μηνύματα
\item Τελικά υποβάλλει δύο μηνύματα $ M_0, M_1 $  στο σύστημα, διαφορετικά από αυτα που μπορει να αποκρυπτογραφήσει
\item Το σύστημα διαλέγει τυχαία 1 bit $b$ και αποστέλλει το $C=Enc(M_b)$ στον $\adv$
\item Ο $\adv$ μπορεί να συνεχίσει να κρυπτογραφεί πολυωνυμικό πλήθος μηνυμάτων και να κάνει οποιονδηποτε υπολογισμό θέλει
\item Προαιρετικά ο $\adv$ μπορει να συνεχίσει να χρησιμοποιεί το \textit{decryption oracle} 
\item Τελικά μαντεύει το $b$
\end{itemize}


\begin{block}{Ορισμός ασφάλειας}
Το κρυπτοσύστημα έχει την ιδιότητα IND-CCA1 αν κάθε PPT $\adv$ έχει αμελητέο πλεονέκτημα στον υπολογισμό του $b$ από το να μαντέψει τυχαία.

Αν ισχύει το προαιρετικό βήμα το κρυπτοσύστημα είναι IND-CCA2 (adaptive IND-CCA)
\end{block}

\begin{block}{Malleability:Μια σχετική ιδιότητα}
Οποιαδηποτε αλλαγή στο ciphertext οδηγεί σε αντίστοιχη αλλαγή στο plaintext. Κάποιες φορές είναι επιθυμητή και κάποιες όχι.
\end{block}

\end{frame}

\begin{frame}[allowframebreaks]{Παραδείγματα με \textit{παραδοσιακό} RSA}
\begin{block}{To παραδοσιακό RSA δεν είναι IND-CPA γιατί είναι deterministic}
Αν τα πιθανά μηνύματα είναι:
\begin{itemize}
\item $m_1$ = "Buy IBM"
\item $m_2$ = "Sell IBM"
\end{itemize}
τότε ο $\adv$ μπορεί να τα κρυπτογραφήσει και να τα συγκρίνει με το νόμιμο ciphertext
\end{block}

\begin{block}{To παραδοσιακό RSA είναι malleable}
\begin{itemize}
\item Στόχος: Αλλοίωση του $c = m^e \md{n}$
\item $c' = c (\frac{9}{10})^e \md{n} = (m \frac{9}{10})^e \md{n}$
\item H αποκρυπτογράφηση δίνει το $m \frac{9}{10}$
\item Ο $\adv$ μπορεί να \textit{αλλοιώσει} κάποιο μήνυμα χωρίς να το γνωρίζει
\end{itemize}
\end{block}

\begin{block}{To παραδοσιακό RSA δεν είναι IND-CCA}
Ο $\adv$ μπορεί να αποκρυπτογραφήσει μηνύματα επιλογής του
\begin{itemize}
\item Στόχος: Αποκρυπτογράφηση του $c = m^e \md{n}$
\item Μπορεί να αποκρυπτογραφήσει το $ c'= c x^e \md{n}$ όπου το $x$ είναι δικής του επιλογής
\item Ανακτά το $m = \frac{m'}{x}$ 
\end{itemize}
\end{block}

\end{frame}

\begin{frame}[allowframebreaks]{Παραδείγματα με \textit{παραδοσιακό} ElGamal}

To ElGamal είναι IND-CPA αν ισχύει η DDH assumption

\begin{block}{To παραδοσιακό El Gamal είναι malleable}
\begin{itemize}
\item Στόχος: Αλλοίωση του $c = (G,M) = (g^r, m h^r)$
\item $c' = (G',M') = (G g^{r'}, M \frac{9}{10} h^{r'}) = (g^{r+r'}, m \frac{9}{10} h^{r+r'}) $
\item H αποκρυπτογράφηση $\frac{M'}{G'^x} $ δίνει το $m \frac{9}{10}$
\item Ο $\adv$ μπορεί να \textit{αλλοιώσει} κάποιο μήνυμα χωρίς να το γνωρίζει
\end{itemize}
\end{block}

\begin{block}{To παραδοσιακό El Gamal δεν είναι IND-CCA}
Ο $\adv$ μπορεί να αποκρυπτογραφήσει μηνύματα επιλογής του
\begin{itemize}
\item Στόχος: Αποκρυπτογράφηση του $c = (G,M) = (g^r, m h^r)$
\item $c' = (G',M') = (G g^{r'}, M a h^{r'}) = (g^{r+r'}, m a h^{r+r'}) $ όπου το $a$ επιλέγεται από τον $\adv$
\item H αποκρυπτογράφηση $\frac{M'}{G'^x} $ δίνει το $am$ και κατά συνέπεα το $m$
\item Ο $\adv$ μπορεί να \textit{αλλοιώσει} κάποιο μήνυμα χωρίς να το γνωρίζει
\end{itemize}
\end{block}

\end{frame}

\begin{frame}[allowframebreaks]{Λύσεις RSA}
\begin{block}{Randomised Encryption}
\begin{itemize}
\item Αντί για κρυπτογράφηση $m$ κρυπτογράφηση $f(m,r)$ όπου $r$ random
\item Η $f$ είναι εύκολα αντιστρέψιμη από οποιονδήποτε
\item Μια απλή υλοποίηση της $f$: random padding
\item Χρήση στο SSL μέχρι πρόσφατα: PKCS1
\end{itemize}
\end{block}

\begin{block}{Η επίθεση του Bleichenbacher (1998) \cite{Bleichenbacher98chosenciphertext}}
\begin{itemize}
\item Στόχος: Αποκρυπτογράφηση του $c = f(m,r)^e \md{n}$
\item Αποστολή πολλών μηνυμάτων της μορφής $ c'= c x^e \md{n}$ με τυχαια $x$
\item O $\adv$ προσπαθεί να βρει μηνύματα $m'$ για τα οποία $f(m',r) = (c')^d \md{n}$
\item Ανακτά το $m = \frac{m'}{x}$ 
\item Πρακτικά: χρήση $SSL$ error codes ως decryption oracle
\item Με 300.000 εως 2.000.000 $c'$ μπορεί να αποκρυπτογραφηθεί το $c$
\item \textbf{Λύση}: RSA - OAEP secure in the random oracle model \cite{Bellare95optimalasymmetric}
\end{itemize}
\end{block}
 
\end{frame}

\begin{frame}[allowframebreaks]{Λύσεις El Gamal:Cramer Shoup cryptosystem \cite{cs98}}
\begin{itemize}
\item Ronald Cramer, Victor Shoup, Crypto 1998
\item Επέκταση του El Gamal
\item Χρηση συνάρτησης σύνοψης $H$
\item Αν ισχυει η υπόθεση DDH, τότε παρέχει IND-CCA2
\end{itemize}

\begin{block}{Δημιουργία Κλειδιών}
\begin{itemize}
\item Επιλογή πρώτων $p,q$ με $p=2q+1$
\item $G$ ειναι η υποομάδα ταξης $q$ στον $\zns{p}$
\item Επιλογή random generators $g_1, g_2$

\item Επιλογή τυχαίων στοιχείων $x_1,x_2,y_1,y_2,z \in \zn{q}$
\item $c=g_1^{x_1}g_2^{x_2} d=g_1^{y_1}g_2^{y_2},h=g_1^{z}$
\item Δημόσιο Κλειδί: $(c,d,h)$
\item Μυστικό Κλειδί: $(x_1,x_2,y_1,y_2,z)$
\end{itemize}
\end{block}

\begin{block}{Κρυπτογράφηση}
\begin{itemize}
\item Μετατροπή μηνύματος $m$ στο $G$
\item Επιλογή τυχαίου $r \in \zn{q}$
\item Υπολογισμός
\begin{itemize} 
\item $u_1 = g_1^r,u_2 = g_2^r$
\item $e = m h^r$
\item $\alpha = H(u_1,u_2,e)$
\item $v = c^r d^{r\alpha}$ 
\end{itemize}
\item Κρυπτογράφημα: $(u_1,u_2,e,v)$
\end{itemize}
\end{block}

\begin{block}{Αποκρυπτογράφηση}
\begin{itemize}
\item Υπολογισμός $\alpha = H(u_1,u_2,e)$
\item Έλεγχος αν $u_1^{x_1}u_2^{x_2}(u_1^{y_1}u_2^{y_2})^\alpha=v$. Σε περίπτωση αποτυχίας έξοδος χωρίς αποκρυπτογράφηση
\item Σε περιπτωση επιτυχίας υπολογισμός $m = \frac{e}{u_1^z}$
\end{itemize}
\end{block}

\begin{block}{Παρατηρήσεις}
\begin{itemize}
\item $h,z$ αντιστοιχούν σε δημόσιο - ιδιωτικό κλειδί  El Gamal
\item $u_1, e$ αντιστοιχούν στο κρυπτογράφημα του El Gamal
\item H $H$ μπορεί να αντικατασταθεί για αποφυγή του random oracle
\item $u_2,v$ λειτουργεί ως έλεγχος ακεραιότητας, ώστε να  μπορεί να αποφευχθεί το malleability 
\item Διπλάσια πολυπλοκότητα από ElGamal τόσο σε μέγεθος κρυπτοκειμένου, όσο και σε υπολογιστικές απαιτήσεις
\end{itemize}
\end{block}
\end{frame}

\begin{frame}[allowframebreaks]{References}
\begin{small}
\nocite{*}
\bibliographystyle{alpha}
\bibliography{attacks}
\end{small}
\end{frame}

\end{document}