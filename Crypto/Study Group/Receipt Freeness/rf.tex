\documentclass{beamer}
\usepackage{amsmath,amssymb,amsthm,array}
\usepackage{bm}
\usepackage{xltxtra}
\usepackage{multirow}
\usepackage{multicol}
\usepackage{algorithm}
\usepackage{algorithmic}
\usetheme{CambridgeUS}
\usecolortheme{wolverine}
\usefonttheme{serif}
\setmainfont[Mapping=TeX-text]{GFS Neohellenic}
\setbeamertemplate{navigation symbols}{}
\title{Receipt Freeness In Voting Systems}
\author{Panagiotis Grontas}
\date{06/12/2013}
\defbeamertemplate*{footline}{shadow theme}
{%
  \leavevmode%
  \hbox{\begin{beamercolorbox}[wd=.5\paperwidth,ht=2.5ex,dp=1.125ex,leftskip=.3cm plus1fil,rightskip=.3cm]{author in head/foot}%
    \usebeamerfont{author in head/foot}\insertframenumber\,/\,\inserttotalframenumber\hfill\insertshortauthor (\insertshortinstitute)
  \end{beamercolorbox}%
  \begin{beamercolorbox}[wd=.5\paperwidth,ht=2.5ex,dp=1.125ex,leftskip=.3cm,rightskip=.3cm plus1fil]{title in head/foot}%
    \usebeamerfont{title in head/foot}\insertshorttitle%
  \end{beamercolorbox}}%
  \vskip0pt%
}
\institute{$\mu\Pi\lambda\forall$  - CoReLab Crypto Group}


\setlength{\columnseprule}{0.4pt}
\begin{document}

\begin{frame}
\titlepage
\end{frame}

\begin{frame}{Context}

\begin{itemize}
\item \textbf{Voting Privacy}: Leak no information about voting, to enable secret ballot elections
\item Privacy: During and after voting
\item\textbf{Voting Verifiability}: Leak enough information to enable everybody to check the result
\item \textbf{Facts}:
\begin{itemize}
	\item Voters might want to reveal their votes after the elections
	\item Verification information might aid them to
\end{itemize}
\item Privacy should be enforced
\item Privacy and verifiability are contradicting
\item Verifiability without usable evidence
\end{itemize}

\end{frame}

\begin{frame}[allowframebreaks]{Receipt Freeness \cite{BT94}}
\begin{block}{Receipt-Freeness}
The voter cannot prove how he voted after the fact

The protocol must maintain verifiability
\end{block}

\begin{itemize}
\item In traditional voting schemes the voter is isolated during the voting process
\item He brings nothing outside of the voting booth
\item An electronic analogue is needed
\begin{block}{Definition}
A voting booth is a component that provides information theoretically secure communication between two entities. 
\end{block}
\item Voting is done in \textit{booth mode}
\item The voter can forge all messages he receives.
\item It is not enough
\end{itemize}

\begin{block}{E-Voting Schemes always yield a receipt}
\begin{itemize}
\item Public Key Encryption is randomised to be secure
\item The randomisation can server as a receipt
\item A coercer might use it to check if the voter cooperated or not
\item Simply encrypt the selection using the randomness and check the BB for the ciphertext
\end{itemize}
\end{block}

\framebreak

\begin{block}{Main idea}
\begin{itemize}
\item Adaptation of the first verifiable voting scheme \cite{CF85},\cite{CY86}
\item The voter does not construct the ballot
\item Instead, it is constructed by the authorities
\item The voter simply selects his choice, without providing any input
\item Every action is proved
\end{itemize}
\end{block}

\framebreak

\begin{enumerate}
\item \textbf{The authority} generates $\eta+1$ ballots as randomised encryptions of a $yes-vote$ and a $no-vote$ in random order.
\item In \textit{booth mode}, the voter receives the decryption of a specific component for each of the $\eta+1$ ballots.
\item The authority proves publicly that the ballots are of the correct form as in \cite{CF85}.
\begin{itemize}
\item The beacon generates $\eta$ random bits $c_i$
\item $c_i=0$: the current ballot is decrypted with a proof of correctness
\item $c_i=1$: Prove that the master ballot contains valid encryptions by connecting the current ballot with the master ballot
\end{itemize}
\item The voter \textit{publicly indicates} the preferred index in the first ballot, to be his vote.
\end{enumerate}

\framebreak

\begin{itemize}
\item Each authority $A_j$ selects a blinding factor $f_j$ and supplies its encryption $e_j$ to each voter
\item Inside the voting booth, the voter is given $f_j$ and proof that its encryption is $e_j$ (using the beacon)
\item The voter selects the vote $v_i$ and a polynomial $P_i$ in order to share the votes to the authorities. $P_i(0)=v_i$
\item Each share is blinded by calculating $y_j = P(j) - f_j$
\item Each voter releases the pair $(y_j,e_j)$ for all authorities. This tuple is the actual vote
\item The authorities add the blinded shares and multiply the encryptions for all the voters. Since the encryption is homomorphic the product of the $e_j$ can be decrypted and used to unblind the sum of the blinded shares
\item Each authority now has a point for the combined voter polynomial $Q = \sum_{i=1}^N P_i$ where $Q(0) = \sum_{i=1}^N P_i(0)$ yields the election result 
\item At least $t$ authorities can reconstruct the polynomial and retrieve the tally
\end{itemize}

\framebreak

\cite{BT94} Receipt Freeness was rebutted in \cite{HS00}
\begin{itemize}
\item Turn the validity argument around
\item Commit to an ordering of the test ballots $(v_0,v_1,...,v_\eta)$
\item The receipt is the commitment
\item Each beacon bit validates the receipt
\item Probability of receipt validity = Probability of vote validity = $1-2^{-\eta}$
\end{itemize}

\end{frame}

\begin{frame}[allowframebreaks]{Adding Receipt Freeness to Mixnets \cite{SK95}}

\begin{definition}{Chameleon Bit Commitments (Chaum, Brassard and Crepeau)}
A bit commitment that can open in both 0 and 1 using a trapdoor
\end{definition}

\begin{definition}
An untappable channel is an addon to the voter that provides information theoretically secure one way communication to another party. 
\end{definition}

\begin{block}{Requirement}
The channels used by the decommitments are \textit{physically untappable}.
\end{block}

\begin{itemize}
\item \cite{SK95}: the first verifiable mixnet
\item ...providing receipt freeness
\item The encrypted votes are not be prepared by the voter, but by the system
\item The system must convince the voter of the validity of the encryption in a way that is not transferable to a coercer.
 

\item \cite{SK95} utilises the mixnet \textit{backwards}
\begin{itemize}
\item The last mix server prepares encryption of  a pair of \textit{yes/no} votes in random internal order for each voter.
\item He shuffles and reneencrypts the lists and proves the correctness of the action using the cut and choose protocol.
\item He then sends the list to the next (previous) mix server, which is actually the previous one
\item Each mix server rearranges randomly the internal order of the votes and proceeds to his normal reencryption and permutation
\end{itemize}

\framebreak

\item The voter must pinpoint the encryption destined for him in the shuffled list and choose internally his preferred option, either yes or no.
\item He must know the internal reorderings, the shuffles and the reencryptions. 
\item Each mix server must commit to these actions and send the commitments to the voter
\item The opening of the commitments will provide the necessary information to the voter.
\item It must be done however in a manner that does not provide a receipt - \textit{chameleon blobs}
\item In order to vote, the voter selects the correct component and uses the mixnet in the normal forward fashion. 
\end{itemize}

\end{frame}

\begin{frame}[allowframebreaks]{Designated Verifier Proofs \cite{JSI96}}

\begin{itemize}
\item ZK proofs offer public verifiability
\item ... The \textit{coercer} can verify too ...
\item The proof can serve as a receipt
\item Ability to forge a proof is needed
\item Proofs verifiable only by a designated verifier
\end{itemize}

\begin{block}{Idea}
 
\begin{itemize}
\item The prover wants to prove statement $\Phi$ to the verifier
\item Construct \textit{verifier-specific version} $\Phi \bigvee DV$ where \textit{DV=I am the designated verifier}
\item $DV$ corresponds to knowledge or possession of some secret trapdoor information on a chameleon commitment scheme
\item Only the designated verifier has access to the trapdoor information (a secret key).
\item If the proof is valid then $\Phi$ is valid
\item Transfer the proof: $DV$ is true so the disjunction is always true
\end{itemize}
\end{block}

\framebreak

\begin{block}{Implementation using Pedersen commitment}
\begin{itemize}
\item Let $x_V, y_V=g^{x_v}$ be a secret public key pair.
\item The prover commits to $w$ by computing $g^w y_V^{r}$
\item To open the commitment the prover sends $(w,r)$ 
\item The verifier can validate the commitment since he knows $x_V$
\item The verifier can present to an adversary $(w',r')$ such that the commitment is valid
\end{itemize}
\end{block}

\end{frame}


\begin{frame}[allowframebreaks]{Adding Receipt Freeness to Homomorphic Schemes \cite{HS00}}

\begin{block}{Idea}
\begin{itemize}
\item Combine \cite{SK95} with \cite{CGS97}
\item Free the voter from the need to encrypt his choice
\item The voting authorities prepare encryptions of the possible votes and the voter simply picks the choice of interest
\item Maintain ballot secrecy using a re-encryption mixnet.
\item Notify voter of all the permutations and reenecryption factors used, to enable him to pinpoint   the vote of choice
\item Use of designated verifier proofs
\item Requirement: \textit{untappable one way communication channels} from the authorities to the voters
\end{itemize}
\end{block}

\framebreak 

\begin{itemize} 
\item \textbf{Vote Generation}
\begin{enumerate}
\item Each valid choice $v_i \in \{ v_1, v_2, \cdots v_C \}$ is deterministically encrypted by the public key encryption algorithm (using predetermined randomness) producing the list of the encryptions of the valid choices $L_0=\{e_1^0, \cdots, e_C^0\}$. Everybody can validate these encryptions by encrypting each choice with the randomness agreed.
\item Each authority $A_k$ reencrypts and permutes $L_{k-1}$ thus producing $L_k$
\item Each authority \textit{publicly} proves that each vote in $L_{k-1}$ has a reencryption in $L_k$.
\item Each authority sends using the untappable channel the permutation $\pi_k$ it applied and \textit{privately} proves that it is valid.
\end{enumerate}
\item \textbf{Vote Casting} Using the permutation information received from each authority the voter tracks down the output item that encrypts his choice.
The vote casted is the index of the desired item.
\item \textbf{Vote Tallying} Since the scheme applies to homomorphic cryptosystems, the authorities compute the homomorphic function and decrypt the tally, while proving its correctness. 
\end{itemize}

\framebreak

\begin{block}{WID Proof of reencryption \textit{(for verifiability)}} 
\textbf{Input} An encrypted vote $(x,y)$ and a list of encrypted votes $L=\{(x_i,y_i)\}_{i=1}^L$

\textbf{Output} Proof that there exists a reencryption of $(x,y)$ in L, without revealing the position

\textbf{Private Input for prover} $t$ the index of the reencryption $(x_t,y_t) = (g^{\xi}x,h^{\xi}y)$ 
\end{block}

\begin{itemize}
\item The prover randomly selects $\{ d_i \}_{i=1}^L, \{ r_i \}_{i=1}^L$ 
\item He calculates $\{ a_i = (\frac{x_i}{x})^{d_i} g^{r_i} \}_{i=1}^C$ and $\{ b_i = (\frac{y_i}{y})^{d_i} h^{r_i} \}_{i=1}^C$. 
\item For the actual reencrypted value the prover calculates $a_t = g^{\xi d_t + r_t}$ and $b_t = h^{\xi d_t + r_t}$. 
\item The prover offers the calculated values to the  verifier.
\item The verifier randomly selects a challenge $c$ and returns it to the prover.
\item The prover recalculates $d_t$ as $d_t' = c - \sum_{i=1, i \neq t}^C d_i$ and finds $r_t'$ such that $\xi d_t + r_t = \xi d_t' + r_t'$. 
\item Then he updates the values of $d_t,r_t$ to $d_t',r_t'$. 
\item Finally he submits the values $\{ d_i \}_{i=1}^C, \{ r_i \}_{i=1}^C$ to the verifier.
\item The verifier validates that:
\begin{itemize}
\item $c=\sum_i d_i$
\item $\{ a_i = \frac{x_i}{x}^{d_i} g^{r_i} \}_{i=1}^C$
\item $\{ b_i = \frac{y_i}{y}^{d_i} h^{r_i} \}_{i=1}^C$.
\end{itemize}
\end{itemize}
 
\framebreak

\textbf{DV Proof of reencryption}
\begin{block}{Ρeceipt Freeness}
Prove that $(x',y')$ is a reencryption of $(x,y)$ with witness $\xi$ 

Equivalently $(x',y')=(g^\xi x,h^\xi y)$

The verifier has a (public,secret) key pair $(h_{ver},z_{ver})$. 

Use the private key in order to forge the proof for the coercer, while maintaining its validity. 
\end{block}



\begin{itemize}
\item The prover randomly selects $d,w,r$ and sends to the verifier the values 
\begin{itemize}
\item 	$a=g^d$
\item  	$b=h^d$
\item 	$s=g^w h_{ver}^r=g^{w+z_{ver}r}$
\end{itemize}  
\item The verifier selects a random challenge $c$ and sends it to the prover.
\item The prover calculates $u=d+\xi(c+w)$ and sends $u,w,r$ to the verifier.
\item The verifier validates that: 
\begin{itemize}
\item $s=g^w h_{ver}^r$
\item $g^u = \frac{x'}{x}^{c+w} a$
\item $h^u = \frac{y'}{y}^{c+w} b$
\end{itemize}
\end{itemize}

\begin{block}{How to forge the proof}
\begin{itemize}
\item The verifier knows the secret key 
\item Find $w',r'$ such that $w+z_{ver}r=w'+z_{ver}r'$ 
\item Forge the proof for any $(x'',y'')$, by randomly selecting the offers $a,b$ and calculating $w'=a-c$ and $r'=\frac{b-w}{z_{ver}}$
\end{itemize}
\end{block}
\end{frame}

\begin{frame}[allowframebreaks]{References}
\begin{small}
\nocite{*}
\bibliographystyle{alpha}
\bibliography{rf}
\end{small}
\end{frame}
 
\end{document}
