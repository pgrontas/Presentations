\section{Ζεύξεις και Διγραμμικές Απεικονίσεις}

Στην ενότητα αυτή θα ασχοληθούμε με τις ζεύξεις, οι οποίες αποτελούν ένα πολύ σημαντικό σύγχρονο κρυπτογραφικό εργαλείο με πολλές εφαρμογές.

\subsection{Εισαγωγή}

Μία ζεύξη είναι μία συνάρτηση η οποία αντιστοιχεί στοιχεία από μία ομάδα πηγή $\mathcal G$ σε μία ομάδα προορισμό $\mathcal G_T$. Το χαρακτηριστικό τους που μας ενδιαφέρει στην Κρυπτογραφία είναι ότι ενώ στο $\mathcal G$ κάποια προβλήματα είναι δύσκολα, μπορούν να γίνουν εύκολα  μέσω της ζεύξης στο $\mathcal G_T$. Συγκεκριμένα:
\index{Ζεύξη}
\begin{definition}
Μία ζεύξη (pairing) είναι μία αποδοτικά υπολογίσιμη συνάρτηση $e : \mathcal G \times \mathcal G \rightarrow \mathcal G_T$ η οποία είναι:
\begin{itemize}
\item Διγραμμική (bilinear): $e(g^a,g^b) = e(g,g)^{ab}$ όπου $g \in \mathcal G \ a,b \in \mathbb{Z}$
\item Μη εκφυλισμένη (non-degenerate): Αν $\mathcal G=<g>$ τότε $\mathcal G_T = <e(g,g)>$
\end{itemize}
\end{definition}

Ένα τέτοιο παράδειγμα αποτελεί η αντιστοίχιση σημείων ελλειπτικής καμπύλης σε ένα πεπερασμένο σώμα (δηλ. $\mathcal E(GF(p)) \times \mathcal E(GF(p)) \rightarrow GF(p^a))$
Από τον παραπάνω ορισμό προκύπτει ότι οι ζεύξεις είναι `συμμετρικές':
\begin{equation}
e(g^a,h^b) = e(g^b,h^a) = e(g,h)^{ab}
\end{equation}

To πρόβλημα \gls{DLOG} (\ref{def:DLOG}) μπορεί να αναχθεί από το $\mathcal G$ στο $\mathcal G_T$. Πράγματι αν θέλουμε να υπολογίσουμε τον διακριτό λογάριθμο $x$ ενός στοιχείου $y=g^x$ μπορούμε να το κάνουμε ελέγχοντας τις ζεύξεις $e(g,y)$ και $e(g,g)^x$. Στην μία ομάδα όμως αυτό μπορεί να είναι πιο εύκολο από την άλλη, κάτι που είχε ανησυχήσει αρχικά πολλούς κρυπτογράφους.

Σε κάθε περίπτωση οι ζεύξεις καθιστούν τo \gls{DDH} (\ref{DDHP}) εύκολα υπολογίσιμο. Για να δούμε αν $g^c = g^{ab}$ μπορούμε να υπολογίζουμε (αποδοτικά) το $e(g^a, g^b) = e(g,g)^{ab}$ και να το συγκρίνουμε με το $e(g,g^c)=e(g,g)^c$

Αυτό σημαίνει ότι το \gls{DDH} πρέπει να αντικατασταθεί. Για τον σκοπό αυτό έχει οριστεί το \emph{διγραμμικό ανάλογο του} ως εξής:
\index{Πρόβλημα!Απόφασης Diffie-Hellman!Διγραμμικό}
\begin{definition}{Διγραμμικό Πρόβλημα Απόφασης των Diffie-Hellman (\gls{BDDH})} 
\label{BDDHP}
\begin{center}
\emph{Δίνονται}: δύο στοιχεία $h,g \in \mathcal G$ και τα στοιχεία $g^\alpha, g^\beta,  e(h,g)^c$.\\
\emph{Ζητείται}: Ισχύει $c = \alpha \beta$;
\end{center}
\end{definition}

Μπορεί να αποδειχθεί ότι $BDDHP \Rightarrow DDHP$ στο $\mathcal G_T$ (άσκηση \ref{ask12-2}).


\subsection{Τριμερής Ανταλλαγή Κλειδιού}

Μία από τις πρώτες εφαρμογές των ζεύξεων στην κρυπτογραφία δόθηκε από τον A. Joux στο \cite{Joux00} και αφορούσε την ανταλλαγή κλειδιών Diffie Hellman μεταξύ τριών οντοτήτων. 

Αν υποθέσουμε ότι δουλεύουμε σε μία κυκλική ομάδα με γεννήτορα $g$ και οι τρεις οντότητες $A,B,C$ έχουν ζευγάρια ιδιωτικών - δημοσίων κλειδιών $(x_A, y_A=g^{x_A}),(x_B, y_B=g^{x_B}),(x_C, y_C=g^{x_C})$. Μπορεί να συμφωνηθεί ένα κοινό κλειδί μεταξύ τους ως εξής:

Αυτό μπορεί να γίνει σε τρεις γύρους ως εξής:
\begin{enumerate}
\item Ο $A$ στέλνει το $y_A$ στον $B$, ο $B$ στέλνει το $y_B$ στον $C$, ο $C$ στέλνει το $y_C$ στον $A$ (κυκλικά).
\item Ο $A$ υπολογίζει το $t_A = y_C^{x_A} = g^{x_Cx_A}$, o $B$ υπολογίζει το $t_B = y_A^{x_B} = g^{x_Bx_A}$ και ο $C$ υπολογίζει το $t_C = y_B^{x_C} = g^{x_Bx_C}$
\item Ο $A$ στέλνει το $t_A$ στον $B$, ο $B$ στέλνει το $t_B$ στον $C$, ο $C$ στέλνει το $t_C$ στον $A$ (πάλι κυκλικά).
\item Όλοι υπολογίζουν το κοινό κλειδί ως εξής:
\begin{itemize}
	\item Ο $A$ με $t_C ^ {x_A} = g^{x_Bx_Cx_A}$ 
	\item Ο $B$ με $t_A ^ {x_B} = g^{x_Cx_Ax_B}$
	\item Ο $C$ με $t_B ^ {x_C} = g^{x_Ax_Bx_C}$
\end{itemize}
\end{enumerate}

Στο \cite{Joux00} προτάθηκε το ίδιο πρωτόκολλο με ένα γύρο ανταλλαγής μηνυμάτων χρησιμοποιώντας ζεύξεις. 
Υποθέτουμε δύο ομάδες  $\mathcal G_1$, $\mathcal G_2$ με τάξη ένα πρώτο $p$ και μία συμμετρική διγραμμική ζεύξη $e : \mathcal G_1 \times \mathcal G_1 \rightarrow \mathcal G_2$. 
\index{Πρωτόκολλο ανταλλαγής κλειδιού!Diffie-Hellman!Τριμερές Με Ζεύξεις}
\begin{itemize}
\item Όλοι οι συμμετέχοντες εκπέμπουν τα δημόσια κλειδιά τους $y_A=g^{x_A},y_B=g^{x_B},y_C=g^{x_C}$.
\item Με την βοήθεια της ζεύξης το κοινό κλειδί μπορεί να υπολογιστεί ως εξής:
\begin{itemize}
\item $e(g^{x_B},g^{x_C})^{x_A} = e(g,g)^{x_Bx_Cx_A}$
\item $e(g^{x_A},g^{x_C})^{x_B} = e(g,g)^{x_Ax_Cx_B}$
\item $e(g^{x_A},g^{x_B})^{x_C} = e(g,g)^{x_Ax_Bx_C}$
\end{itemize}
\end{itemize}

\subsection{Εφαρμογές}

Η εργασία του Joux προκάλεσε το ενδιαφέρον των κρυπτογράφων για τις ζεύξεις και οδήγησε σε αρκετές εφαρμογές τους. 

\subsubsection{Σύντομες υπογραφές διακριτού λογαρίθμου}
Η πρώτη τέτοια εφαρμογή προτάθηκε από τους Boheh, Lynn και Shacham  στο \cite{BLS04}. Προσπαθεί να παρουσιάσει υπογραφές που βασίζονται στο \gls{DLOG}, αλλά έχουν μικρότερο μέγεθος από αυτές που αναφέραμε στις ενότητες \ref{fig:ElGamalDS} και \ref{sec:DSS} οι οποίες αποτελούνται από δύο ή περισσότερους ακέραιους μεγέθους όσο η τάξη της επιλεγμένης ομάδας. Συγκεκριμένα οι υπογραφές του \cite{BLS04} έχουν μέγεθος όσο ένας ακέραιος και είναι συγκρίσιμες με τις υπογραφές \gls{RSA} (\ref{fig:RSADS}). 

Με απλά λόγια το σχήμα υπογραφών BLS λειτουργεί ως εξής:
\begin{itemize}
\item Οι συμμετέχοντες συμφωνούν σε μία διγραμμική απεικόνιση $e$ μεταξύ των ομάδων $(\mathcal G,\mathcal G_T)$ όπου το \gls{CDH} είναι δύσκολο στο $\mathcal G$. Υποθέτουμε πως το $\mathcal G$ παράγεται από το $G$ και έχει τάξη $n$
\item Το ιδιωτικό κλειδί του αποστολέα είναι ένας ακέραιος $a \in [1,n-1]$ ενώ το δημόσιο είναι το σημείο $A = aG$ (με προσθετικό  συμβολισμό).
\item Για να παραχθεί η υπογραφή σε ένα μήνυμα $m$ παράγεται η σύνοψη $M=\hash(m)$ και υπολογίζεται το στοιχείο $S=aM$ το οποίο είναι και η υπογραφή.
\item Για την επαλήθευση αρκεί να ελεγχθεί ότι το $\mathcal G,M,A,S$ ικανοποιεί το DDHP το οποίο όπως είπαμε είναι εύκολο μέσω της ζεύξης. Αρκεί να ελέγξουμε δηλαδή αν $e(\mathcal G,A)=e(M,S)$
\end{itemize}

Από την άλλη η πλαστογράφηση παραμένει δύσκολη καθώς πρέπει να  λυθεί το \gls{CDH} στο $\mathcal G$.

\subsection{Κρυπτογράφηση με βάση την ταυτότητα}
\label{sec:IBE}
\index{Κρυπτογραφία!Με Βάση Την Ταυτότητα}
Όπως είδαμε στην ενότητα \ref{sec:PKI} για να λειτουργήσει πρακτικά η κρυπτογραφία δημοσίου κλειδιού απαιτεί την ύπαρξη μίας υποδομής για την διανομή των δημοσίων κλειδιών και κυρίως εμπιστοσύνη σε αυτή την υποδομή.  Προτού ακόμα εφαρμοστούν οι πρώτες υποδομές δημοσίων κλειδιών ο Shamir \cite{ShamirIBE84} πρότεινε μια εναλλακτική λύση, την \emph{κρυπτογράφηση με βάση την ταυτότητα (\gls{IBE})}. Η βασική ιδέα του σχήματος αφορά την χρήση των ταυτοτήτων των χρηστών για δημόσια κλειδιά.

Συγκεκριμένα, θεωρούμε πως, κάθε χρήστης ενός \gls{IBE} κρυπτοσυστήματος διαθέτει μία ταυτότητα, όπως για παράδειγμα μιας διεύθυνσης ηλεκτρονικού ταχυδρομείου. Μία έμπιστη τρίτη οντότητα αναλαμβάνει να δημιουργήσει το ιδιωτικό κλειδί από την ταυτότητα και να το παραδώσει ασφαλώς στον χρήστη. Οποιοσδήποτε χρήστης θέλει να κρυπτογραφήσει ένα μήνυμα, το κάνει με βάση την ταυτότητα του παραλήπτη και το δημόσιο κλειδί της έμπιστης αρχής (ακόμα και πριν την δημιουργία του  ιδιωτικού κλειδιού).

Η πρώτη πρακτική υλοποίηση \gls{IBE} δόθηκε από τους Boneh και Franklin στο \cite{BonehIBE01} μέσω ζεύξεων. Λειτουργεί ως εξής:
\begin{itemize}
\item Οι συμμετέχοντες συμφωνούν σε μία διγραμμική απεικόνιση $e$ μεταξύ των ομάδων $(\mathcal G,\mathcal G_T)$ όπου το \gls{BDDH} είναι δύσκολο. Υποθέτουμε πως το $\mathcal G$ παράγεται από το $G$ και έχει τάξη $n$. Επίσης υποθέτουμε δύο συναρτήσεις σύνοψης $\hash_{\mathcal G}, \hash$ στο $\mathcal G$ και στο $\MSG$ αντίστοιχα.
\item Η έμπιστη αρχή έχει ως ιδιωτικό κλειδί το $t \in [1,\cdots, n-1]$ και δημόσιο κλειδί το $T=tG$.
\item Για την παραγωγή του ιδιωτικού κλειδιού χρησιμοποιείται η συνάρτηση σύνοψης $\hash_{\mathcal G}$ και παράγεται το $x = t \hash_{\mathcal G}(ID)$
\item Για την κρυπτογράφηση ενός μηνύματος προς τον χρήστη $ID$, o αποστολέας:
\begin{itemize}
\item υπολογίζει αρχικά το $Y=\hash_{\mathcal G}(ID)$
\item Όπως και στο Elgamal επιλέγεται ένα τυχαίο $r \in [1,n-1]$ και υπολογίζει το $R=rG$. Στην συνέχεια εφαρμόζεται η ζεύξη και υπολογίζεται το $C = m \xor \hash(e(Y,T)^r)$ 
\item To κρυπτοκείμενο είναι το ζεύγος $(R,C)$.
\end{itemize}
\item H αποκρυπτογράφηση γίνεται χρησιμοποιώντας το ιδιωτικό κλειδί και τη ζεύξη $m=c \xor \hash(e(x,R))$
\end{itemize}

Η ορθότητα του κρυπτοσυστήματος βασίζεται στη διγραμμικότητα της ζεύξης:
\begin{center}
$e(x,R) = e(tY,rG) = e(Y,G)^{rt} = e(Y,tG)^r = e(Y,T)^r$
\end{center}

Η ασφάλεια του κρυπτοσυστήματος βασίζεται στο \gls{BDDH}. 

O προσεκτικός αναγνώστης θα προσέξει ότι η \gls{IBE} έχει και αυτή ανάλογα προβλήματα με το PKI, όπως για παράδειγμα τη διανομή του δημοσίου κλειδιού της έμπιστης αρχής ή την ασφαλή διανομή των ιδιωτικών κλειδιών στους χρήστες. Μια αναλυτική σύγκριση μπορεί να βρεθεί στο \cite{paterson2003comparison}.
